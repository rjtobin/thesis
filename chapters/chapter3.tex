\chapter{The Spectral Radius of Outerplanar and Planar Graphs}
\section{Introduction}

The study of spectral radius of planar graphs has a long history, dating back to at least Schwenk
and Wilson \cite{SchwenkWilson1978}.  This direction
of research was further motivated by applications where the spectral
radius is used as a measure of the connectivity of a network, in
particular for planar
networks in areas such as geography, see for example \cite{BootsRoyle1991}
and its references.  To compare connectivity of networks to a theoretical upper bound, geographers were
interested in finding the planar graph of maximum spectral radius. To this end, Boots and Royle and
independently Cao and Vince conjectured that the extremal graph is $P_2 + P_{n-2}$
\cite{BootsRoyle1991}, \cite{CaoVince1993}. Several researchers have worked on this problem and
successively improved upon the best theoretical upper bound, including \cite{Hong1988},
\cite{CaoVince1993}, \cite{Hong1995}, \cite{Guiduli1996}, \cite{Hong1998},
\cite{EllinghamZha2000}. Other related problems have been considered, for example Dvo\v{r}\'{a}k
and Mohar found an upper bound on the spectral radius of planar graphs
with a given maximum degree \cite{DvorakMohar2010}. Work has also been done maximizing the spectral radius of graphs on surfaces of higher genus \cite{EllinghamZha2000, Hong1995, Hong1998}. We would also like to note that it is claimed in \cite{EllinghamZha2000} that Guiduli and Hayes proved
that the maximum spectral radius of a planar graph is attained by $P_2 + P_{n-2}$,
for sufficiently large $n$. However, this preprint has never appeared, and the authors
could not be reached for comment on it.


The outerplanar conjecture appeared
in \cite{CvetkovicRowlinson1990}, where the authors mention that it is related
to the study of various subfamilies of Hamiltonian graphs.
Rowlinson \cite{Rowlinson1990} made partial progress on this conjecture, which
was also worked on by Cao and Vince \cite{CaoVince1993} and Zhou--Lin--Hu \cite{ZhouLinHu2001}.

% TODO: introduce method by giving proof of Mantel

% TODO: introduce notation

\section{Outerplanar Graphs of Maximum Spectral Radius}\label{outerplanar}
Let $G$ be a graph. As before, let the first eigenvector of the adjacency matrix of $G$ be $\textbf{v}$ normalized so that maximum entry is $1$. For $v\in V(G)$ we will use $v$ to mean a vertex or the eigenvector entry of that vertex, where it will be clear from context which meaning we are using. Let $x$ be a vertex with maximum eigenvector entry, ie $x=1$.  Throughout let $G$ be an outerplanar graph on $n$ vertices with maximal adjacency spectral radius.  $\lambda_1$ will refer to $\lambda_1(A(G))$. 

Two consequences of $G$ being outerplanar that we will use frequently are that $G$ has at most $2n-3$ edges and $G$ does not contain $K_{2,3}$ as a subgraph. An outline of our proof is as follows. We first show that there is a single vertex of large degree and that the remaining vertices have small eigenvector entry (Lemma \ref{small eigvec}). We use this to show that the vertex of large degree must be adjacent to every other vertex (Lemma \ref{bound bad elements}). From here it is easy to prove that $G$ must be $K_1+P_{n-1}$.

\begin{figure}[]
\begin{center}
\begingroup

\setlength{\unitlength}{.01cm}
{
\setlength{\fboxsep}{10pt}
\framebox[1.5\width]{
\begin{tikzpicture}[rotate=90, scale=1.3]


\coordinate (W) at (1.553380,0.70606);

\coordinate (W1) at ($(W) + 0*(0,-0.7)$);
\coordinate (W2) at ($(W) + 1*(0,-0.7)$);
\coordinate (W3) at ($(W) + 2*(0,-0.7)$);
\coordinate (W4) at ($(W) + 3*(0,-0.7) + (0,-0.25)$);
\coordinate (W5) at ($(W) + 4*(0,-0.7) + (0,-0.25)$);
\coordinate (W6) at ($(W) + 5*(0,-0.7) + (0,-0.25)$);


\coordinate (B) at ($0.5*(W3) + 0.5*(W4) + (-1.0,0)$);


\draw (W1) -- (W2);
\draw (W2) -- (W3);
\draw [dashed] (W3) -- (W4);
\draw (W4) -- (W5);
\draw (W5) -- (W6);

\draw (B) -- (W1);
\draw (B) -- (W2);
\draw (B) -- (W3);
\draw (B) -- (W4);
\draw (B) -- (W5);
\draw (B) -- (W6);

\filldraw[blue] (W) circle (0.07cm);
\filldraw[blue] (W1) circle (0.07cm);
\filldraw[blue] (W2) circle (0.07cm);
\filldraw[blue] (W3) circle (0.07cm);
\filldraw[blue] (W4) circle (0.07cm);
\filldraw[blue] (W5) circle (0.07cm);
\filldraw[blue] (W6) circle (0.07cm);

\filldraw[blue] (B) circle (0.07cm);

\end{tikzpicture}}
}
\endgroup
\end{center}
\caption{The graph $P_1 + P_{n-1}$.
   \label{fig:outerplanar}}
\end{figure}

We begin with an easy lemma that is clearly not optimal, but suffices for our needs.
\begin{lemma}\label{trivial lambda bound}
 $\lambda_1 > \sqrt{n-1}$.
\end{lemma}
\begin{proof}
 The star $K_{1,n-1}$ is outerplanar, and cannot be the maximal outerplanar
 graph with respect to spectral radius because it is a strict subgraph of other outerplanar graphs on the same vertex set.  Hence, $\lambda_1(G) > \lambda_1(K_{1,n}) = \sqrt{n-1}$.
\end{proof}

\begin{lemma}
 For any vertex $u$, we have $d_u > \mathbf{v}_un - 11\sqrt{n}$.
\end{lemma}
\begin{proof}
 Let $A$ be the neighborhood of $u$, and let $B = V(G) \setminus (A\cup \{u\})$.  We have 
  \[ \lambda_1^2 \mathbf{v}_u = \sum_{y \sim u} \sum_{z \sim y} \mathbf{v}_z \leq d_u + \sum_{y \sim u} \sum_{z \in N(y)\cap A} \mathbf{v}_z + \sum_{y \sim u} \sum_{z \in N(y)\cap B} z. \]

\noindent By outerplanarity, each vertex in $A$ has at most two neighbors in $A$, otherwise
$G$ would contain a $K_{2,3}$.  In particular,
 \[ \sum_{y \sim u} \sum_{z \in N(y)\cap A} \mathbf{v}_z  \leq 2 \sum_{y \sim u} \mathbf{v}_y = 2\lambda_1 \mathbf{v}_u. \]
Similarly, each vertex in $B$ has at most $2$ neighbors in $A$.  So
 \[ \sum_{y \sim u} \sum_{z \in N(y)\cap B} \mathbf{v}_z \leq 2 \sum_{z \in B} \mathbf{v}_z \leq \frac{2}{\lambda_1} \sum_{z \in B} d_z \leq \frac{4e(G)}{\lambda_1} \leq \frac{4(2n-3)}{\lambda_1}, \]
as  $e(G) \leq 2n-3$ by outerplanarity.  So, using Lemma~\ref{trivial lambda bound} we have
 \[ \sum_{y \sim u} \sum_{z \in N(y)\cap B} \mathbf{v}_z < 8 \sqrt{n}.\]
Combining the above inequalities yields
 \[ \lambda_1^2 \mathbf{v}_u - 2\lambda_1 \mathbf{v}_u < d_u + 8 \sqrt{n}.\]
Again using Lemma~\ref{trivial lambda bound} we get
 \[ \mathbf{v}_u n - 11\sqrt{n} <  (n-1 - 2\sqrt{n-1}) \mathbf{v}_u - 8 \sqrt{n} < d_u .\]
\end{proof}

\begin{lemma}\label{small eigvec}
 We have $d_x > n - 11 \sqrt{n}$ and for every other vertex $u$, $\mathbf{v}_u < C_1 / \sqrt{n}$ for some absolute constant $C_1$, for $n$ sufficiently large.
\end{lemma}
\begin{proof}
The bound on $d_x$ follows immediately from the previous lemma and the normalization that
$\mathbf{v}_x=1$.  Now consider any other vertex $u$.  We know that $G$ contains no $K_{2,3}$, so $d_u < 12 \sqrt{n}$, otherwise $u$ and $x$ share $\sqrt{n}$ neighbors, which
yields a $K_{2,3}$ if $n \geq 9$.  So
 \[ 12 \sqrt{n} > d_u > \mathbf{v}_u n - 11\sqrt{n}, \]
that is, $\mathbf{v}_u < 23 / \sqrt{n}$.
\end{proof}

\begin{lemma}\label{bound bad elements}
 Let $B = V(G) \setminus (N(x) \cup \{x\})$.  Then
  \[ \sum_{z \in B} \mathbf{v}_z < C_2 / \sqrt{n} \]
 for some absolute constant $C_2$.
\end{lemma}
\begin{proof}
From the previous lemma, we have $|B| < 11 \sqrt{n}$.  Now
 \[ \sum_{z \in B} \mathbf{v}_z \leq \frac{1}{\lambda_1} \sum_{z \in B} \left(23 / \sqrt{n}\right) d_z = \frac{23}{\lambda_1 \sqrt{n}} \left( e(A,B) + 2 e(B)\right) . \]
Each vertex in $B$ is adjacent to at most two vertices in $A$, so $e(A,B) \leq 2 |B| < 22 \sqrt{n}$.  The graph induced on $B$ is outerplanar, so
$e(B) \leq 2|B| - 3 < 22 \sqrt{n}$.  Finally, using the fact that $\lambda_1 > \sqrt{n-1}$, we get the required result.
\end{proof}

\begin{theorem}
 For sufficiently large $n$, $G$ is the graph $K_1 + P_{n-1}$, where $+$ represents the graph join operation.
\end{theorem}
\begin{proof}
First we show that the set $B$ above is empty, i.e. $x$ is adjacent
to every other vertex.  If not, let $y \in B$.  Now $y$ is adjacent to at
most two vertices in $A$, and so by Lemma~\ref{small eigvec} and Lemma~\ref{bound bad elements}, 
 \[ \sum_{z \sim y} \mathbf{v}_z < \sum_{z \in B} \mathbf{v}_z + 2 C_1 / \sqrt{n} < (C_2 + 2 C_1) / \sqrt{n} < 1\]
when $n$ is large enough.  Let $G^+$ be the graph obtained
from $G$ by deleting all edges incident to $y$ and replacing them by the single edge $\{x,y\}$.  The resulting graph is outerplanar.  Then,
using the Rayleigh quotient,
 \[ \lambda_1(A^+) - \lambda_1(A) \geq \frac{\textbf{v}^t(A^+ - A)\textbf{v}}{\textbf{v}^t\textbf{v}} = \frac{2\mathbf{v}_y}{\textbf{v}^t\textbf{v}} \left(1 - \sum_{z \sim y} \mathbf{v}_z\right) > 0.\]
This contradicts the maximality of $G$.  Hence $B$ is empty.


Now $x$ is adjacent to every other vertex in $G$.  Hence every vertex other than $x$ has degree less than or equal to $3$.  Moreover, the graph induced by 
$V(G) \setminus \{x\}$ cannot contain any cycles, as then $G$ would not be outerplanar.
It follows that $G$ is a subgraph of $K_1 + P_{n-1}$, and maximality ensures that $G$ must be equal to $K_1 + P_{n-1}$.
\end{proof}


\section{Planar Graphs of Maximum Spectral Radius}\label{planar}

\subsection{Structural Lemmas}

As before, let $G$ be a graph with first eigenvector normalized so that maximum entry is $1$, and let $x$ be a vertex with maximum eigenvector entry, ie $x=1$. Let $m = |E(G)|$. For subsets $X, Y\subset V(G)$ we will write $E(X)$ to be the set of edges induced by $X$ and $E(X,Y)$ to be the set of edges with one endpoint in $X$ and one endpoint in $Y$. We will let $e(X,Y) = |E(X,Y)|$. We will often assume $n$ is large enough without saying so explicitly. Throughout the section, let $G$ be the planar graph on $n$ vertices with maximum spectral radius, and let $\lambda_1$ denote this spectral radius.

We will use frequently that $G$ has no $K_{3,3}$ as a subgraph, that $m\leq 3n-6$, and that any bipartite subgraph of $G$ has at most $2n-4$ edges. The outline of our proof is as follows. We first show that $G$ has two vertices that are adjacent to most of the rest of the graph (Lemmas \ref{planar trivial spectral bound}--\ref{second vertex of large degree}). We then show that the two vertices of large degree are adjacent (Lemma \ref{x connected to w}), and that they are adjacent to every other vertex (Lemma \ref{A empty}). The proof of the theorem follows readily.

\begin{figure}[]
\begin{center}
\begingroup

\setlength{\unitlength}{.01cm}
{
\setlength{\fboxsep}{10pt}
\framebox[1.5\width]{
\begin{tikzpicture}[rotate=90, scale=1.3]

\coordinate (U) at (0.553380,1.150606);
\coordinate (V) at (2.553380,1.150606);
\coordinate (W) at (1.553380,0.70606);

\coordinate (W1) at ($(W) + 0*(0,-0.7)$);
\coordinate (W2) at ($(W) + 1*(0,-0.7)$);
\coordinate (W3) at ($(W) + 2*(0,-0.7)$);
\coordinate (W4) at ($(W) + 3*(0,-0.7)$);
\coordinate (W5) at ($(W) + 4*(0,-0.7)$);
\coordinate (W6) at ($(W) + 5*(0,-0.9)$);

\draw (U) -- (V);
\draw (U) -- (W);
\draw (V) -- (W);

\draw (W1) -- (W2);
\draw (U) -- (W2);
\draw (V) -- (W2);

\draw (W2) -- (W3);
\draw (U) -- (W3);
\draw (V) -- (W3);

\draw (W3) -- (W4);
\draw (U) -- (W4);
\draw (V) -- (W4);

\draw (W4) -- (W5);
\draw (U) -- (W5);
\draw (V) -- (W5);

\draw [dashed] (W5) -- (W6);
\draw (U) -- (W6);
\draw (V) -- (W6);

\filldraw[blue] (U) circle (0.07cm);
\filldraw[blue] (V) circle (0.07cm);
\filldraw[blue] (W) circle (0.07cm);
\filldraw[blue] (W1) circle (0.07cm);
\filldraw[blue] (W2) circle (0.07cm);
\filldraw[blue] (W3) circle (0.07cm);
\filldraw[blue] (W4) circle (0.07cm);
\filldraw[blue] (W5) circle (0.07cm);
\filldraw[blue] (W6) circle (0.07cm);

\end{tikzpicture}}
}
\endgroup
\end{center}
\caption{The graph $P_2 + P_{n-2}$.
   \label{fig:planar}}
\end{figure}


\begin{lemma}\label{planar trivial spectral bound}
$\sqrt{6n} > \lambda_1> \sqrt{2n-4}$.
\end{lemma}
\begin{proof}
For the lower bound, first note that the graph $K_{2,n-2}$ is planar and is a strict subgraph of some other planar graphs on the same vertex set. Since $G$ has maximum spectral radius among all planar graphs on $n$ vertices,
\[
 \lambda_1 > \lambda_1(K_{2,n-2}) = \sqrt{2n-4}.
 \]
 For the upper bound, since the sum of the squares of the eigenvalues equals twice the number of edges in $G$, which is
 at most $6n-12$ by planarity, we get that $\lambda_1 < \sqrt{6n-12} < \sqrt{6n}$.
\end{proof}

%We never use all of this averaging right? TK
%By the eigenvector equation, we see
%\[
%\lambda^2 = \sum_{y\sim x}\sum_{z\sim y} z \leq \sum_{uv\in E(G)} (u+v) - \sum_{y\sim x} y = \sum_{uv\in E(G)} (u+v) - \lambda.
%\]
%Rearranging and dividing by $2|E(G)|$ we have
%\[
%\frac{1}{2m} (\lambda^2+\lambda) \leq \frac{1}{2m} \sum_{uv\in E(G)} u+v.
%\]
%Because $G$ is planar, $m\leq 2n-6$, and so we have that the average over all $uv\in E(G)$ of the quantity $\frac{u+v}{2}$ is 
%\begin{equation}\label{average edge one third}
%\mathrm{avg}_{uv\in E(G)} \frac{u+v}{2} \geq \frac{2n-4 + \sqrt{2n-4}}{6n-12} > \frac{1}{3}.
%\end{equation}

Next we partition the graph into vertices of small eigenvector entry and those with large eigenvector entry.  Fix $\epsilon > 0$,
whose exact value will be chosen later.  Let 
\[
L:= \{\mathbf{v}_z\in V(G): \mathbf{v}_z> \epsilon\}
\]
and $S = V(G) \setminus L$. For any vertex $z$, equation~\eqref{eigenvector equation} gives $\mathbf{v}_z\sqrt{2n-4} < \mathbf{v}_z\lambda_1\leq d_z$. Therefore,
\[
2(3n - 6)  \geq \sum_{z\in V(G)} d_z \geq \sum_{z\in L} d_z \geq |L|\epsilon \sqrt{2n-4},
\]
yielding $|L| \leq \frac{3\sqrt{2n-4}}{\epsilon}$. Since the subgraph of $G$ consisting of edges with one endpoint in $L$ and one endpoint in $S$ is a bipartite planar graph, we have $e(S,L) \leq 2n-4$, and since the subgraphs induced by $S$ and by $L$ are each planar, we have $e(S) \leq 3n-6$ and $e(L) \leq \frac{9\sqrt{2n-4}}{\epsilon}$. 

%Similarly to before
%\[
%\lambda^2 + \lambda_1\leq \sum_{uv\in E(G)} u+v = \sum_{uv\in E(S,L)} u+v + \sum_{uv\in E(S)} u+v + \sum_{uv\in E(L)} u+v \leq \sum_{uv\in E(S,L)} (u+v )+ 6\epsilon n + \frac{18\sqrt{n}}{\epsilon}.
%\]
%Dividing both sides by $2e(S,L)$, we have that for $n$ large enough
%\[
%\frac{(2-7\epsilon)n}{2e(S,L)} \leq \frac{1}{e(S,L)}\sum_{uv\in E(S,L)} \frac{u+v}{2}.
%\]
%Since $e(S,L) \leq 2n-4$, we have that the average over $uv\in E(S,L)$ of the quantity $\frac{u+v}{2}$ satisfies 
%\begin{equation}\label{average bipartite edge one half}
%\mathrm{avg}_{uv\in E(S,L)} \frac{u+v}{2} > \frac{1}{2} - 2\epsilon.
%\end{equation}

Next we show that there are two vertices adjacent to most of $S$. The first step towards this is an upper bound on the sum of eigenvector entries in both $L$ and $S$.
\begin{lemma}
\begin{equation}\label{eigenvector norm L}
\sum_{z\in L} \mathbf{v}_z \leq  \epsilon \sqrt{2n-4} + \frac{18}{\epsilon}
\end{equation}
\noindent and
 \begin{equation}\label{eigenvector norm S}
 \sum_{z\in S} \mathbf{v}_z \leq (1+3\epsilon)\sqrt{2n-4}.
 \end{equation}

\end{lemma}
\begin{proof}
\[
\sum_{z\in L} \lambda_1 \mathbf{v}_z = \sum_{z\in L} \sum_{y\sim z} \mathbf{v}_y = \sum_{z\in L}\left( \sum_{\substack{y\sim z \\ y\in S}} \mathbf{v}_y + \sum_{\substack{y\sim z\\ y\in L}} \mathbf{v}_y \right) \leq \epsilon e(S,L) + 2e(L) \leq \epsilon (2n-4) + \frac{18\sqrt{2n-4}}{\epsilon}.
\]
Dividing both sides by $\lambda_1$ and using Lemma \ref{planar trivial spectral bound} gives \eqref{eigenvector norm L}.

 On the other hand,
 \[
 \sum_{z\in S} \lambda_1 \mathbf{v}_z = \sum_{z\in S}\sum_{y\sim z} \mathbf{v}_y \leq 2\epsilon e(S) + e(S,L) \leq (6n-12)\epsilon  + (2n-4).
 \]
 Dividing both sides by $\lambda_1$ and using Lemma \ref{planar trivial spectral bound} gives \eqref{eigenvector norm S}.
 \end{proof}

 Now, for $u\in L$ we have
\[
\mathbf{v}_u\sqrt{2n-4} \leq \lambda_1 \mathbf{v}_u=\sum_{y\sim u} \mathbf{v}_y = \sum_{\substack{y\sim u \\ y\in L}} \mathbf{v}_y + \sum_{\substack{y\sim u\\ y\in S}} \mathbf{v}_y \leq \sum_{y\in L} \mathbf{v}_y + \sum_{\substack{y\sim u \\y\in S}} \mathbf{v}_y.
\]
By \eqref{eigenvector norm L}, this gives
\begin{equation}\label{eigenvector norm neighbors of w}
\sum_{\substack{y\sim u \\ y\in S}} \mathbf{v}_y \geq (\mathbf{v}_u-\epsilon)\sqrt{2n-4} - \frac{18}{\epsilon}.
\end{equation}

 
 The equations \eqref{eigenvector norm S} and \eqref{eigenvector norm neighbors of w} imply that if $u\in L$ and $\mathbf{v}_u$ is close to $1$, then the sum of the eigenvector entries of vertices in $S$ not adjacent to $u$ is small. The following lemma is used to show that $u$ is adjacent to most vertices in $S$.
 
\begin{lemma}\label{eigenvector entry lower bound}
For all $z$ we have $\mathbf{v}_z>\frac{1}{\sqrt{6n}}$.
\end{lemma}
\begin{proof}
  By way of contradiction assume $\mathbf{v}_z\leq \frac{1}{\sqrt{6n}} < \frac{1}{\lambda_1}$. By
  equation~\eqref{eigenvector equation} $z$ cannot be adjacent to $x$, since
  $x$ has eigenvector entry $1$. Let $H$ be the graph obtained from $G$ by removing all edges incident with $z$ and making $z$ adjacent to $x$. Using the Rayleigh quotient, we have $\lambda_1(H) > \lambda_1(G)$, a contradiction.
\end{proof}

\noindent Now letting $u=x$ and combining \eqref{eigenvector norm neighbors of w} and \eqref{eigenvector norm S}, we get 
\[
(1+3\epsilon)\sqrt{2n-4} \geq \sum_{\substack{y\in S\\ y\not\sim x}} \mathbf{v}_y + \sum_{\substack{y\in S\\ y\sim x}} \mathbf{v}_y \geq \sum_{\substack{y\in S\\ y\not\sim x}} \mathbf{v}_y + (1-\epsilon)\sqrt{2n-4} - \frac{18}{\epsilon}.
\]
Now applying Lemma \ref{eigenvector entry lower bound} gives
\[
|\{y\in S: y\not\sim x\}| \frac{1}{\sqrt{6n} }\leq 4\epsilon \sqrt{2n-4} + \frac{18}{\epsilon}.
\]
For $n$ large enough, we have $|\{y\in S: y\not\sim x\}| \leq 14\epsilon n$. So $x$ is adjacent to most of $S$. Our next goal is to show that there is another vertex in $L$ that is adjacent to most of $S$.

\begin{lemma}\label{second vertex of large degree}
There is a $w\in L$ with $w\not=x$ such that $\mathbf{v}_w> 1-24\epsilon$ and $|\{y\in S: y\not\sim w\}| \leq 94\epsilon n$.
\end{lemma}
\begin{proof}
By equation~\eqref{eigenvector equation}, we see
\[
\lambda_1^2 = \sum_{y\sim x}\sum_{z\sim y} \mathbf{v}_z \leq \left(\sum_{uv\in E(G)} \mathbf{v}_u+\mathbf{v}_v\right) - \sum_{y\sim x} \mathbf{v}_y = \left(\sum_{uv\in E(G)} \mathbf{v}_u+\mathbf{v}_v\right) - \lambda_1.
\]
Rearranging and noting that $e(S) \leq 3n-6$ and $e(L) \leq \frac{9\sqrt{2n-4}}{\epsilon}$ since $S$ and $L$ both induce planar subgraphs gives
\begin{align*}
& 2n-4\leq \lambda_1^2 + \lambda_1\leq \sum_{uv\in E(G)} \mathbf{v}_u+\mathbf{v}_v = \left(\sum_{uv\in E(S,L)} \mathbf{v}_u+\mathbf{v}_v\right) + \left(\sum_{uv\in E(S)} \mathbf{v}_u+\mathbf{v}_v\right) + \left(\sum_{uv\in E(L)} \mathbf{v}_u+\mathbf{v}_v \right) \\
& \leq \left(\sum_{uv\in E(S,L)} \mathbf{v}_u+\mathbf{v}_v \right) + \epsilon(6n-12) + \frac{18\sqrt{2n-4}}{\epsilon}.
\end{align*}
So for $n$ large enough,
\[
(2-7\epsilon)n \leq \sum_{uv\in E(S,L)} \mathbf{v}_u+\mathbf{v}_v =\left( \sum_{\substack{uv\in E(S,L)\\ u=x}} \mathbf{v}_u+\mathbf{v}_v\right) + \left(\sum_{\substack{uv\in E(S,L)\\ u\not= x}} \mathbf{v}_u+\mathbf{v}_v\right) \leq \epsilon e(S,L) + d_x + \sum_{\substack{uv\in E(S,L)\\ u\not= x} } \mathbf{v}_u,
\]
giving
\[
\sum_{\substack{uv\in E(S,L)\\ u\not = x}} \mathbf{v}_u \geq (1-9\epsilon)n.
\]

Now since $d_x \geq |S| - 14\epsilon n > (1-15\epsilon)n$, and $e(S,L) < 2n$, the number of terms in the left hand side of the sum is at most $(1+15\epsilon)n$. By averaging, there is a $w\in L$ such that 
\[
\mathbf{v}_w \geq \frac{1-9\epsilon}{1+15\epsilon} > 1-24\epsilon .
\]
Applying \eqref{eigenvector norm neighbors of w} and \eqref{eigenvector norm S} to this $w$ gives
\[
(1+3\epsilon) \sqrt{2n-4} \geq \sum_{\substack{y\in S\\ y\not\sim w}} \mathbf{v}_y + \sum_{\substack{y\in S\\ y\sim w}} \mathbf{v}_y \geq \sum_{\substack{y\in S\\ y\not\sim w}} \mathbf{v}_y + (1-21\epsilon)\sqrt{2n-4} + \frac{18}{\epsilon},
\]
and applying Lemma \ref{eigenvector entry lower bound} gives that for $n$ large enough
\[
|\{y\in S: y\not\sim w\}| \leq 94\epsilon n .
\]

\end{proof}
\medskip

In the rest of the section, let $w$ be the vertex from Lemma \ref{second vertex of large degree}. So $\mathbf{v}_x=1$ and $\mathbf{v}_w> 1-24\epsilon$, and both are adjacent to most of $S$. Our next goal is to show that the remaining vertices are adjacent to both $x$ and $w$. Let $B = N(x) \cap N(w)$ and $A = V(G) \setminus \{x\cup w \cup B\}$. We show that $A$ is empty in two steps: first we show the eigenvector entries of vertices in $A$ are as small as we need, which we then use to show that if there is a vertex in $A$ then $G$ is not extremal.

\begin{lemma}\label{eigenvector entries of A small}
Let $v\in V(G) \setminus \left\{ x,w \right\}$. Then $\mathbf{v}_v < \frac{1}{10}$.
\end{lemma}

\begin{proof}
We first show that the sum over all eigenvector entries in $A$ is small, and then we show that each eigenvector entry is small. Note that for each $v\in A$, $v$ is adjacent to at most one of $x$ and $w$, and is adjacent to at most $2$ vertices in $B$ (otherwise $G$ would contain a $K_{3,3}$ and would not be planar). Thus
\[
\lambda_1\sum_{v\in A} \mathbf{v}_v \leq \sum_{v\in A} d_v \leq 3|A| +2e(A) < 9|A|,
\]
where the last inequality holds by $e(A) < 3|A|$ since $A$ induces a planar graph. Now, since $|L| < \frac{3\sqrt{2n-4}}{\epsilon} < \epsilon n$ for $n$ large enough, we have $|A| \leq (14+94+1)\epsilon n$ (by Lemma \ref{second vertex of large degree}) . Therefore 
\[
\sum_{v\in A} \mathbf{v}_v \leq \frac{9\cdot 109 \cdot \epsilon n}{\sqrt{2n-4}}.
\]

Now any $v\in V(G) \setminus \left\{ x,w \right\}$ is adjacent to at most 4 vertices in $B \cup \left\{x,w \right\}$, as otherwise we would have a $K_{3,3}$
as above.  So we get
\[
\lambda_1\mathbf{v}_v = \sum_{u\sim v} \mathbf{v}_u \leq 4 + \sum_{\substack{u\sim v\\ u\in A}} \mathbf{v}_u \leq 4 + \sum_{u\in A} \mathbf{v}_u \leq C\epsilon\sqrt{n},
\]
where $C$ is an absolute constant not depending on $\epsilon$. Dividing both sides by $\lambda_1$ and choosing $\epsilon$ small enough yields the result.
\end{proof}

We use the fact that the eigenvector entries in $A$ are small to show that if $v\in A$ (i.e. $v$ is not adjacent to both $x$ and $w$), then removing all edges from $v$ and adding edges from it to $x$ and $w$ increases the spectral radius, showing that $A$ must be empty. To do this, we must be able to add edges from a vertex to both $x$ and $w$ and have the resulting graph remain planar. This is accomplished by the following lemma.

\begin{lemma}\label{x connected to w}
If $G$ is extremal, then $x\sim w$.
\end{lemma}

Once $x\sim w$, one may add a new vertex adjacent to only $x$ and $w$ and the resulting graph remains planar.

\begin{proof}[Proof of Lemma \ref{x connected to w}]
From above, we know that for any $\delta > 0$, we may choose $\epsilon$ small enough so that when $n$ is sufficiently
large we have $d_x > (1-\delta)n$ and $d_w > (1-\delta)n$.  By maximality
of $G$, we also know that $G$ has precisely $3n-6$ edges, and by Euler's formula, any planar drawing of $G$ has $2n-4$ faces, each of which is bordered by precisely three edges of $G$ (because in a maximal planar graph, every face is a triangle).    


Now we obtain a bound on the number of faces that $x$ and $w$ must be incident to.  Let $X$ be the set of edges incident to $x$.  Each edge in $G$ is incident to precisely two faces, and each face can be incident to at most two edges in $X$ (again, since each face is a triangle by maximality).  So $x$ is incident to at least $|X| = d_x \geq (1-\delta)n$ faces.  Similarly, $w$ is incident to at least $(1-\delta)n$ faces. 


Let $F_1$ be the set of faces that are incident to $x$, and then let $F_2$ be the set of faces that are not incident to $x$, but which share an edge with a face in $F_1$.  Let $F = F_1 \cup F_2$.  We have $|F_1| \geq (1-\delta)n$.  Now each
face in $F_1$ shares an edge with exactly three other faces:  if two faces shared two edges, then since each face is a triangle
both faces must be bounded by the same three edges;  this cannot happen, except in the degenerate case when $n=3$.  At most two of these three faces are in $F_1$, and so $|F_2| \geq |F_1| / 3 \geq (1-\delta)n / 3$.  Hence, $|F| \geq (1-\delta)4n / 3$, and so the sum of the number of faces
in $F$ and the number of faces incident to  $w$ is larger than $2n-4$.  In 
particular, there must be some face $f$ that is both belongs to $F$ and is incident to $w$.


Since $f \in F$, then either $f$ is incident to $x$ or $f$ shares an edge with some face that is incident to $x$.  If $f$ is incident to both $x$ and $w$, then $x$ is adjacent to $w$ and 
we are done.  Otherwise, $f$ shares an edge $\left\{y,z\right\}$ with a face $f'$ that is incident to $x$.  In this case, deleting the edge $\left\{y,z\right\}$ and inserting the edge $\left\{x,w\right\}$ yields a planar graph $G'$.  By lemma~\ref{eigenvector entries of A small}, 
the product of the eigenvector entries of $y$ and $z$ is less than $1/100$,
which is smaller than the product of the eigenvector entries of $x$ and $w$.  
This implies that $\lambda_1(G') > \lambda_1(G)$, which is a contradiction.
\end{proof}

We now show that every vertex besides $x$ and $w$ is adjacent to both $x$ and $w$.

\begin{lemma}\label{A empty}
$A$ is empty.
\end{lemma}

\begin{proof}
Assume that $A$ is nonempty. $A$ induces a planar graph, therefore if $A$ is nonempty, then there is a $v\in A$ such that $|N(v)\cap A| < 6$. Further, $v$ has at most $2$ neighbors in $B$ (otherwise $G$ would contain a $K_{3,3}$. Recall that $\textbf{v}$ is the principal eigenvector for the adjacency matrix of $G$. Let $H$ be the graph with vertex set $V(G) \cup \{v'\} \setminus \{v\}$ and edge set $E(H) = E(G\setminus\{v\}) \cup \{v'x, v'w\}$. By Lemma \ref{x connected to w}, $H$ is a planar graph. Then 
\begin{align*}
\textbf{v}^T \textbf{v}\lambda_1(H) &\geq \textbf{v}^T A(H) \textbf{v} & \\
& = \textbf{v}^T A(G) \textbf{v} - 2\sum_{z\sim v} \mathbf{v}_v\mathbf{v}_z + 2\mathbf{v}_v(\mathbf{v}_w+\mathbf{v}_x) & \\
& \geq \textbf{v}^T A(G) \textbf{v} - 14\cdot \mathbf{v}_v \cdot \frac{1}{10} - 2\sum_{\substack{z\sim v\\ z\in \{w,x\}}} \mathbf{v}_v\mathbf{v}_z + 2\mathbf{v}_v(\mathbf{v}_w+\mathbf{v}_x) & \mbox{(by Lemma \ref{eigenvector entries of A small})} \\
& \geq \textbf{v}^T A(G) \textbf{v} - \frac{14}{10}\mathbf{v}_v + 2\mathbf{v}_v\mathbf{v}_w & \mbox{($|N(v)\cap \{x,w\}| \leq 1$)} \\
& > \textbf{v}^T A(G) \textbf{v} & \mbox{(as $\mathbf{v}_w>7/10$)}\\
&= \textbf{v}^T\textbf{v} \lambda_1(G).
\end{align*}
So $\lambda_1(H) > \lambda_1(G)$ and $H$ is planar, i.e. $G$ is not extremal, a contradiction.
\end{proof}


We now have that if $G$ is extremal, then $K_2+I_{n-2}$, the join of an edge and an independent set of size $n-2$, is a subgraph of $G$. Finishing the proof is straightforward.


\subsection{Proof of Main Theorem}

\begin{theorem}
For $n\geq N_0$, the unique planar graph on $n$ vertices with maximum spectral radius is $K_{2} + P_{n-2}$.
\end{theorem}

\begin{proof}
By Lemmas \ref{x connected to w} and \ref{A empty}, $x$ and $w$ have degree $n-1$. We now look at the set $B = V(G) \setminus \{x,w\}$. For $v\in B$, we have $|N(v) \cap B| \leq 2$, otherwise $G$ contains a copy of $K_{3,3}$. Therefore, the graph induced by $B$ is a disjoint union of paths, cycles, and isolated vertices. However, if there is some cycle $C$ in the graph induced by $B$, then $C \cup\{x,w\}$ is a subdivision of $K_5$. So the graph induced by $B$ is a disjoint union of paths and isolated vertices. However, if $B$ does not induce a path on $n-2$ vertices, then $G$ is a strict subgraph of $K_2 + P_{n-2}$, and we would have $\lambda_1(G) < \lambda_1(K_2 + P_{n-2})$. Since $G$ is extremal, $B$ must induce $P_{n-2}$ and so $G = K_2 + P_{n-2}$.
\end{proof}


This chapter is based on part of the paper ``Three conjectures in extremal spectral graph theory'',
 \cite{TaitTobin2017}, to appear in \textit{Journal of Combinatorial Theory, Series B},
written jointly with Michael Tait.  The dissertation
author was the primary investigator and author of the paper.
