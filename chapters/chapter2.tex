\chapter{Measures of Graph Irregularity}
\section{Introduction}

A \textit{measure of graph irregularity} is a  statistics that quantifies how far a graph is
from being regular.  The choice of the word ``measure'' is slightly unfortunate due to its
more common usage in mathematics, but this is the phrase that has been used repeatedly in the
literature, and so we adopt it here.  Many different such statistics have been proposed.
There is the \textit{irregularity of a graph} as defined by Albertson \cite{Albertson1997},
\[ \textrm{irr}(G) = \sum_{(u,v) \in E(G)} \left| d_u - d_v \right| .\]
A variant of this that depends only on the degree sequence of a graph,
the \textit{total irregularity}, was introduced by Abdo et. al. \cite{Abdo2014},
\[ \textrm{irr}_{\textrm{it}}(G) = \frac{1}{2} \sum_{u \in V(G)} \sum_{v \in V(G)} \left| d_u - d_v\right|\]
Another irregularity measure that does depends only on the degree sequence is given by the
variance of the degree sequence, studied by Bell \cite{Bell1992},
\[var(G) = \frac{1}{n} \sum_{v\in V(G)} \left| d_v - \bar{d} \right|^2 . \]
Collatz and Sinogowitz, in perhaps the first spectral graph theory paper, noted
that the difference between the largest adjacency eigenvalue and the average degree
can be seen as a measure of the irregularity of a connected graph \cite{CollatzSinogowitz1957}.
Finally, the \textit{principal ratio} of a connected graph was studied as a
measure of graph irregularity by Cioab\u{a} and Gregory \cite{CioabaGregory2007},
\[ \gamma(G) = \frac{\mathbf{x}_{\text{max}}}{\mathbf{x}_{\text{min}}},\]
where $\mathbf{x}$ is a positive eigenvector corresponding to the
largest eigenvalue of the adjacency matrix, and
$x_{\text{min}}$ and $x_{\text{max}}$ are the smallest and largest eigenvector
entries respectively.

In this section we determine the extremal graphs
with respect to the last two irregularity measures,
answering conjectures of Cioab\u{a} and Gregory \cite{CioabaGregory2007}
and Aouchiche et. al. \cite{AouchicheEtAl2008}.


Let $P_r \cdot K_s$ be the graph attained by identifying an end vertex
of a path on $r$ vertices to any vertex of a complete graph
on $s$ vertices.  This has been called a \textit{kite graph} or a
\textit{lollipop graph}.  Cioab\u{a} and Gregory \cite{CioabaGregory2007}
conjectured that the connected graph on $n$ vertices maximizing $\gamma$
is a kite graph.  Our first result proves this conjecture for $n$ large
enough.
\begin{theorem}\label{main_theorem}
  For sufficiently large $n$, the connected graph $G$ on $n$
  vertices with largest principal ratio is a kite graph.
\end{theorem}


A \textit{pineapple graph} is a clique with pendant edges added to a single vertex.
Aouchiche et al \cite{AouchicheEtAl2008} conjectured that the extremal connected
graph with respect to the invariant $\lambda_1 - d$ is a pineapple graph.  We
show this for sufficiently large $n$.
\begin{theorem}
  For sufficiently large $n$, the connected graph $G$ on $n$
  that maximizes $\lambda_1 - d$ is a pineapple graph.
\end{theorem}



An analogous problem for directed graphs, finding graphs which maximize the
principal ratio for directed graphs, was answered by Aksoy et al \cite{AksoyEtAl2016}.
We note that Brightwell and Winkler \cite{BrightwellWinkler1990} showed that
a kite graph maximizes the expected hitting time of a random walk.
The extremal graphs for various of these irregularity measures have been
studied.  
The extremal graphs with respect to $\textrm{irr}(G)$
were characterized by Hansen and M\'elot \cite{HansenMelot2002},
and the extremal graphs with respect to the total irregularity
were studied by \cite{Abdo2014}.
Nikiforov \cite{Nikiforov2006} proved several inequalities comparing
$var(G)$, $\epsilon(G)$ and $s(G) := \sum_v |d(u) - \bar{d}|$.  
Bell showed that $\epsilon(G)$ and $var(G)$ are incomparable in general
\cite{Bell1992}.  Finally, additional bounds on $\gamma(G)$ have been given in
\cite{CioabaGregory2007, PapendieckRecht2000, Minc1970, Latham1995, Zhang2005}.


\section{Graphs of Maximal Principal Ratio}

\subsection{Structural Lemmas}
%XXX: move this stuff somewhere

Throughout this section $G$ will be a connected simple graph on $n$ vertices.
The eigenvectors and eigenvalues of $G$ are those of the adjacency
matrix $A$ of $G$.  The vector $\mathbf{v}$ will be the eigenvector corresponding
to the largest eigenvalue $\lambda_1$,  and we take $\mathbf{v}$ to be scaled
so that its largest entry is $1$.  Let $x_1$ and $x_k$
be the vertices with smallest and largest eigenvector entries respectively, and
if several such vertices exist then we pick any of them arbitrarily.
Let $x_1, x_2, \cdots, x_k$ be a shortest path between $x_1$ and
$x_k$.  Let $\gamma(G)$ be the principal ratio of $G$.



Recall that the vertices $v_1, v_2, \cdots, v_m$ are a 
\textit{pendant path} if the induced graph on these vertices
is a path and furthermore if, in $G$, $v_1$ has degree $1$ and
the vertices $v_2, \cdots, v_{m-1}$ have degree $2$
(note there is no requirement on the degree of $v_m$).

\begin{lemma}\label{path_bound}
  If $\lambda_1 \geq 2$ and $\sigma = (\lambda_1 + \sqrt{\lambda_1^2 - 4})/2$, then for
  $1 \leq j \leq k$,
   \[ \gamma(G) \leq \frac{\sigma^j - \sigma^{-j}}{\sigma - \sigma^{-1}} \mathbf{v}_{x_j}^{-1}. \]
Moreover we have equality if the vertices $x_1, x_2, \cdots, x_{j}$ are a pendant path.
  
\end{lemma}
\begin{proof}
  We have the following system of inequalities
   \begin{eqnarray*}
     \lambda_1 \mathbf{v}_{x_1} & \geq & \mathbf{v}_{x_2} \\
     \lambda_1 \mathbf{v}_{x_2} & \geq & \mathbf{v}_{x_1} + \mathbf{v}_{x_3} \\
     \lambda_1 \mathbf{v}_{x_3} & \geq & \mathbf{v}_{x_2} + \mathbf{v}_{x_4} \\
     \vdots & & \vdots\\
     \lambda_1 \mathbf{v}_{x_{j-1}} & \geq & \mathbf{v}_{x_j} + \mathbf{v}_{x_{j-2}} .
   \end{eqnarray*}
 The first inequality implies that
  \[ \mathbf{v}_{x_1} \geq \frac{1}{\lambda_1} \mathbf{v}_{x_2}.\]
 Plugging this into the second equation and rearranging gives
  \[ \mathbf{v}_{x_2} \geq \frac{\lambda_1}{\lambda_1^2 - 1} \mathbf{v}_{x_3}. \]
 Now assume that
  \[ \mathbf{v}_{x_i} \geq \frac{{u_{i-1}}}{{u_i}} \mathbf{v}_{x_{i+1}}, \]
 with some positive constants ${u_j}$ for all $j<i$.  Then
  \[ \lambda_1 \mathbf{v}_{x_{i+1}} \geq \mathbf{v}_{x_{i}} + \mathbf{v}_{x_{i+2}} \]
 implies that
 \[ \mathbf{v}_{x_{i+1}} \geq \frac{{u_i}}{\lambda_1 {u_i} - {u_{i-1}}} \mathbf{v}_{x_{i+2}}, \]
 where $\lambda_1 {u_i} - {u_{i-1}}$ must be positive because  $\mathbf{v}_{x_j}$
 is positive for all $j$.
 Therefore the coefficients $u_i$ satisfy the recurrence
  \[ u_{i+1} = \lambda_1 u_i - u_{i-1}\]
 Solving this and using the initial conditions $u_0 = 1$,
 $u_1 = \lambda$ we get
  \[ u_i = \frac{\sigma^{i+1} - \sigma^{-i-1}}{\sigma - \sigma^{-1}} \]
 In particular, $u_i$ is always positive, a fact implicitly
 used above.  Finally this gives,
  \[ \mathbf{v}_{x_1} \geq \frac{u_0}{u_1} \mathbf{v}_{x_2} \geq \frac{u_0}{u_1} \cdot \frac{u_1}{u_2} \mathbf{v}_{x_3} \geq \cdots \geq \frac{\mathbf{v}_{x_j}}{u_{j-1}} \]
 Hence
  \[ \gamma(G) = \frac{\mathbf{v}_{x_k}}{\mathbf{v}_{x_1}} = \frac{1}{\mathbf{v}_{x_1}} \leq \frac{\sigma^j - \sigma^{-j}}{\sigma - \sigma^{-1}} \mathbf{v}_{x_j}^{-1} \]
 If these vertices are a pendant path, then we have equality throughout.
\end{proof}

We will also use the following lemma which comes from the
paper of Cioab\u{a} and Gregory \cite{CioabaGregory2007}.

\begin{lemma}\label{kite_lambda}
  For $r \geq 2$ and $s \geq 3$,
   \[ s - 1 + \frac{1}{s(s-1)} < \lambda_1(P_r \cdot K_s) < s - 1 + \frac{1}{(s-1)^2} . \]
\end{lemma}

In the remainder of the section  we prove Theorem~\ref{main_theorem}.
We now give a sketch of the proof that is contained in Section~\ref{sec_proof}.

\begin{enumerate}
 \item We show that the vertices $x_1, x_2, \cdots, x_{k-2}$ are
   a pendant path and that $x_k$ is connected to all of the vertices
   in $G$ that are not on this path (lemma~\ref{connect_every}).
 \item Next we prove that the length of the path is approximately
   $n - n/\log(n)$ (lemma~\ref{s_range}).
 \item We show that $x_{k-2}$ has degree exactly $2$ (lemma~\ref{k_2_lemma}), which
   extends our pendant path to $x_1, x_2, \cdots, x_{k-1}$.
   To do this, we find conditions under which adding or deleting
   edges increases the principal ratio (lemma~\ref{change}).
 \item Next we show that $x_{k-1}$ also has degree exactly $2$ (lemma~\ref{k_1_lemma}).
   At this point we can deduce that our extremal graph is either
   a kite graph or a graph obtained from a kite graph
   by removing some edges from the clique.  We show that
   adding in any missing edges will increase the principal ratio,
   and hence the extremal graph is exactly a kite graph.
   
\end{enumerate}

\subsection{Proof of Main Theorem}\label{sec_proof}



Let $G$ be the graph with maximal principal ratio among all connected
graphs on $n$ vertices, and let $k$ be the number of vertices in a
shortest path between the vertices with smallest and largest eigenvalue
entries. As above, let $x_1,\cdots, x_k$ be the vertices of the shortest path, where
$\gamma(G) = \mathbf{v}_{x_k} / \mathbf{v}_{x_1}$.  Let $C$ be the set of vertices not on this shortest
path, so $|C| = n-k$.  Note that there is no graph with $n-k=1$, as the endpoints of a path have the same principal eigenvector entry.  Also
$\lambda_1(G) \geq 2$, otherwise $P_{n-2} \cdot K_3$ would have larger
principal ratio.  Finally note that $k$ is strictly larger than $1$,
otherwise $\mathbf{v}_{x_k} = \mathbf{v}_{x_1}$ and $G$ would be regular.


\begin{lemma}\label{max_lambda}
  $\lambda_1(G) > n-k$.
\end{lemma}
\begin{proof}
  Let $H$ be the graph $P_k \cdot K_{n-k+1}$. It is straightforward to see that in $H$, the smallest entry of the principal eigenvector is the vertex of degree $1$ and the largest is the vertex of degree $n-k+1$. Also note that in $H$, the vertices on the path $P_k$ form a pendant path.
  By maximality we know that $\gamma(G) \geq \gamma(H)$.
  Combining this with lemma~\ref{path_bound}, we get
   \[ \frac{\sigma^k - \sigma^{-k}}{\sigma - \sigma^{-1}} \geq \gamma(G) \geq \gamma(H) =  \frac{\sigma_H^{k} - \sigma_H^{-k}}{\sigma_H - \sigma_H^{-1}} \]
  where $\sigma_H = \left(\lambda_1(H) + \sqrt{\lambda_1(H)^2 - 4}\right) /2$.

  \noindent Now the function
   \[ f(x) = \frac{x^k - x^{-k}}{x - x^{-1}}\]
  is increasing when $x \geq 1$.
  Hence we have
  $\sigma \geq \sigma_H$, and so
  $\lambda_1(G) \geq \lambda_1(H) > n-k$.
\end{proof}

\begin{lemma}\label{connect_every}
  $x_1, x_2, \cdots, x_{k-2}$ are a pendant path in $G$, and $x_k$
  is connected to every vertex in $G$ that is not on this
  path.
\end{lemma}
\begin{proof}
  By our choice of scaling, $\mathbf{v}_{x_k} = 1$.  From lemma~\ref{max_lambda}
   \[ n-k < \lambda_1(G) = \sum_{y \sim  x_k} \mathbf{v}_{y} \leq |N(x_k)|. \]
  Now $|N(x_k)|$ is an integer, so we have $|N(x_k)| \geq n-k+1$.
  Moreover because $x_1, x_2, \cdots, x_k$ is an induced path, we
  must have that $|N(x_k)| = n-k+1$ exactly, and hence the
  $N(x_k) = C \cup \{ x_{k-1} \}$.  It follows that $x_1, x_2, \cdots, x_{k-3}$
  have no neighbors off the path, as otherwise there would be a shorter
  path between $x_1$ and $x_k$.
\end{proof}

\begin{lemma}\label{s_range}
For the extremal graph $G$, we have $n-k = (1+o(1))\frac{n}{\log n}$.
\end{lemma}
\begin{proof}
Let $H$ be the graph $P_j \cdot K_{n-j+1}$ where $ j = \left\lfloor n - \frac{n}{\log n}\right\rfloor$, and let $G$ be the connected graph on $n$ vertices with maximum principal ratio. Let $x_1,\cdots, x_k$ be a shortest path from $x_1$ to $x_k$ where $\gamma(G) = \frac{\mathbf{v}_{x_k}}{\mathbf{v}_{x_1}}$. By lemma \ref{connect_every}, we have
\[
\lambda_1(G) \leq \Delta(G) \leq n-k+1.
\]
By the eigenvector equation, this gives that
\begin{equation}\label{gamma of G}
\gamma(G) \leq (n-k+1)^k
\end{equation}
Now, lemma \ref{path_bound} gives that
\[
\gamma(H)  = \frac{\sigma_H^j - \sigma_H^{-j}}{\sigma_H - \sigma_H^{-1}},
\]
where
\[
\sigma(H) = \frac{\lambda_1(H) + \sqrt{\lambda_1(H)^2 -4}}{2}.
\]

Now, $s-1 + \frac{1}{s(s-1)} < \lambda_1(P_r\cdot K_s) < s-1 + \frac{1}{(s-1)^2}$, so we may choose $n$ large enough that $\frac{n}{\log n} + 1 >\sigma_H - \sigma_H^{-1} > \frac{n}{\log n}$. By maximality of $\gamma(G)$, we have
\[
(n-k+1)^k \geq \gamma(G) \geq \gamma(H) \geq \left(\frac{n}{\log n}\right)^{n-\frac{n}{\log n} - 2}.
\]
Thus, $n- k = (1+o(1))\frac{n}{\log n}$.
\end{proof}

For the remainder of this section we will explore the structure of $G$ by
showing that if certain edges are missing, adding them would increase
the principal ratio, and so by maximality these edges must already be
present in $G$.  We have established that the vertices $x_1, x_2, \cdots, x_{k-2}$
are a pendant path, and so we have
\begin{equation}\label{pr_form}
 \gamma(G) = \frac{\sigma^{k-2} - \sigma^{-k+2}}{\sigma-\sigma^{-1}} \frac{1}{\mathbf{v}_{x_{k-2}}}
\end{equation}
We will not add any edges that affect this path, and so the above equality will
remain true.   
The change in $\gamma$ is then completely determined by the change
in $\lambda_1$ and the change in $\mathbf{v}_{x_{k-2}}$.  The next lemma gives conditions
on these two parameters under which $\gamma$ will increase or decrease.

\begin{lemma}\label{change}
Let $x_1, x_2, \cdots, x_{m-1}$ form a pendant path
  in $G$, where $n-m = (1+o(1)) n/\log(n)$.
  Let $G_+$ be a graph obtained from $G$ by adding some edges from $x_{m-1}$ to $V(G)\setminus \{x_1,\cdots, x_{m-1}\}$,
  where the addition of these edges does not affect which vertex
  has largest principal eigenvector entry.  Let
  $\lambda_1^+$ be the largest eigenvalue of $G_+$ with leading eigenvector
  entry for vertex $x$ denoted $\mathbf{v}_x^+$, also normalized to have maximum
  entry one.  
  Define $\delta_1$ and $\delta_2$ such that
  $\lambda_1^+ = (1 + \delta_1) \lambda_1$ and
  $\mathbf{v}_{x_{m-1}}^+ = (1 + \delta_2) \mathbf{v}_{x_{m-1}}$.  Then 
  \begin{itemize}
     \item $\gamma(G_+) > \gamma(G)$ whenever $\delta_1 > 4 \delta_2 / n$
     \item $\gamma(G_+) < \gamma(G)$ whenever $\delta_1 \exp(2 \delta_1 \lambda_1 \log n ) <  \delta_2 / 3 n$.
  \end{itemize}
\end{lemma}
\begin{proof}
  We have
  \[\sigma = \lambda_1 - \lambda_1^{-1} - \lambda_1^{-3} - 2 \lambda_1^{-5} - \cdots - \frac{2}{2n-3} \binom{2n-2}{n} \lambda_1^{-(2n-1)}- \cdots\]
  So
   \[ \lambda_1^+ - \lambda_1 < \sigma_+ - \sigma < \lambda_1^+ - \lambda_1 - 2((\lambda_1^+)^{-1} - \lambda_1^{-1})\]
  when $\lambda_1$ is sufficiently large, which is guaranteed by lemma~\ref{s_range}.
  Plugging in $\lambda_1^+ = (1 + \delta_1) \lambda_1$, we get
   \[ \delta_1 \lambda_1 < \sigma_+ - \sigma < \delta_1 \lambda_1 + 2 \lambda_1^{-1}(1 - (1+\delta_1)^{-1}) < \delta_1 \lambda_1 + \delta_1\]
  In particular
   \[ (1+\delta_1/2) \sigma < \sigma_+ < (1+2 \delta_1) \sigma\]
  To prove part (i), we wish to find a lower bound in the change in the first factor of
  equation~\ref{pr_form}.  Let
   \[ f(x) = \frac{x^{m-1} - x^{-m+1}}{x-x^{-1}}. \]
  Then $2m x^{m-3} > f'(x) > (m-2) x^{m-3} - m x^{m-5}$, and using
  that $n-m \sim  n/\log(n)$ and $\sigma \sim \lambda_1$ which goes to infinity with $n$,
  we get $f'(x) \gtrsim (m-2) x^{m-3}$.  By linearization and because $f(\sigma) \sim \sigma^{m-2}$, it follows that
   \[ \frac{\sigma_+^{m-1} - \sigma_+^{-m+1}}{\sigma_+ - \sigma_+^{-1}} \geq \left(1 + \frac{\delta_1 (m-3)}{2}\right) \frac{\sigma^{m-1} - \sigma^{-m+1}}{\sigma - \sigma^{-1}}\]
  Hence, if
   \[ \frac{\delta_1 (m-3)}{2} > \delta_2\]
  then $\gamma(G_+) > \gamma(G)$.  In particular it
  is sufficient that $\delta_1 > 4\delta_2 / n$.


  To prove part (ii), recall from above that $f'(x) < 2m x^{m-3}$.
  Then, when $x = (1+o(1)) (n / \log(n))$
   \begin{eqnarray*}
     f'(x+\varepsilon) & < & 2m (x+\varepsilon)^{m-3} \\
     & = & 2m x^{m-3} \left( 1 + \frac{\varepsilon}{x} \right)^{m-3}\\
     & \leq & 2m x^{m-3} \exp\left(\frac{m \varepsilon}{x}\right) \\
     & \leq & 2n x^{m-3} \exp(2 \log(n) \varepsilon) 
   \end{eqnarray*}
  So for $0 < \varepsilon < \delta_1 \lambda_1$, we have
   \[ f'(x+\varepsilon) < 2n x^{m-3} \exp(2 \log(n) \delta_1 \lambda_1) \]
  Hence
   \[ \big( 1 + 3n \exp(2 \delta_1 \lambda_1 \log n ) \delta_1 \big) \frac{\sigma^{m-1} - \sigma^{-m+1}}{\sigma - \sigma^{-1}} >  \frac{\sigma_+^{m-1} - \sigma_+^{-m+1}}{\sigma_+ - \sigma_+^{-1}} \]
   
  
\end{proof}

\begin{lemma}\label{large_nbds}
  For every subset of $U$ of $N(x_k)$, we have
   \[ |U| - 1 < \sum_{y \in U} \mathbf{v}_y \leq |U|. \]
  An immediate consequence is that there is at most one
  vertex in the neighborhood of $x_k$ with eigenvector
  entry smaller than $1/2$.
\end{lemma}
\begin{proof}
  The upper bound follows from $\mathbf{v}_y \leq 1$, and the lower bound from
  the inequalities
   \[ \sum_{y \in N(x_k) \setminus U} \mathbf{v}_y  \leq |N(x_k)| - |U|\]
  and
   \[ \sum_{y \in N(x_k)} \mathbf{v}_y = \lambda_1(G) > |N(x_k)| - 1 .\]
\end{proof}

\begin{lemma}\label{k_2_lemma}
 The vertex $x_{k-2}$ has degree exactly $2$ in $G$.
\end{lemma}
\begin{proof}
  Assume to the contrary.  Let $U = N(x_{k-2}) \cap N(x_{k})$.  Then
  $|U| \geq 2$, so by lemma~\ref{large_nbds} we have
   \[ \sum_{y \in U} \mathbf{v}_y > |U| - 1 \geq 1 . \]
  Now, by the same argument as the in the proof of lemma~\ref{path_bound},
  we have that
   \[ \gamma(G) = \frac{\sigma^{k-1} - \sigma^{-k+1}}{\sigma - \sigma^{-1}} \left( \sum_{y \in U} \mathbf{v}_y \right)^{-1} \]
  Let $H = P_{k-1} \cdot K_{n-k+2}$.  Then by maximality of $\gamma(G)$ we have
  \begin{equation*}
   \frac{\sigma^{k-1} - \sigma^{-k+1}}{\sigma - \sigma^{-1}} > \gamma(G) \geq \gamma(H) = \frac{\sigma_H^{k-1} - \sigma_H^{-k+1}}{\sigma_H - \sigma_H^{-1}} 
  \end{equation*}
  So $\sigma > \sigma_H$, which means $\lambda_1(G) > \lambda_1(H) > n-k+1$.
  This means that $\Delta(G) > n-k+1$, but we have established that
  $\Delta(G) = n-k+1$.
\end{proof}

We now know that $x_1, x_2, \cdots, x_{k-1}$ is a pendant path in $G$, and
so equation~\ref{pr_form} becomes
\begin{equation}\label{pr_form2}
 \gamma(G) = \frac{\sigma^{k-1} - \sigma^{-k+1}}{\sigma-\sigma^{-1}} \frac{1}{\mathbf{v}_{x_{k-1}}}
\end{equation}

\begin{lemma}\label{k_1_pre_lemma}
 The vertex $x_{k-1}$ has degree less than $11 |C| / \sqrt{\log n}$.  
\end{lemma}
\begin{proof}
  Assume to the contrary, so throughout this proof we assume that
  the degree of $x_{k-1}$ is at least $11 |C| / \sqrt{\log n}$.
  Let $G_+$ the graph obtained form $G$ with
  an additional edge from $x_{k-1}$ to a vertex $z \in C$ with $\mathbf{v}_z \geq 1/2$.
  Let $\lambda^+_1 = \lambda_1(G_+)$ and  let $\mathbf{v}_{x}^+$ be the principal eigenvector
  entry of vertex $x$ in $G_+$, where this eigenvector is normalized to have
  $\mathbf{v}_{x_{k}}^+ = 1$.


  \noindent \textbf{Change in $\lambda_1$}: By equation~\ref{rayleigh quotient}, we
  have $\lambda_1^+ - \lambda_1 \geq 2 \frac{\mathbf{v}_{x_{k-1}} \mathbf{v}_z}{||\mathbf{v}||^2_2}$.
  A crude upper bound on $||\mathbf{v}||_2^2$ is
   \[ ||\mathbf{v}||_2^2 \leq 1 + \sum_{y \sim x_{k}} \mathbf{v}_y + \frac{2}{\lambda_1} + \frac{4}{\lambda_1^2} + \cdots < 2 \lambda_1 \]
  We also have that $\mathbf{v}_z \geq 1/2$ so
   \[ \lambda_1^+ \geq \left( 1 + \frac{\mathbf{v}_{x_{k-1}}}{2\lambda_1^2} \right) \lambda_1 .\]

  \noindent \textbf{Change in $\mathbf{v}_{x_{k-1}}$}:  Let $U = N(x_{k-1} \cap C)$.
  By the eigenvector equation we have
  \begin{eqnarray*}
    \mathbf{v}_{x_{k-1}} & = & \frac{1}{\lambda_1} \left( \mathbf{v}_{x_{k-2}} + \mathbf{v}_{x_{k}} + \sum_{y \in U} \mathbf{v}_y \right) \\
    \mathbf{v}_{x_{k-1}}^+ & = & \frac{1}{\lambda_1^+} \left( \mathbf{v}_{x_{k-2}}^+ + \mathbf{v}_{x_{k}}^+ + \mathbf{v}_z^+ + \sum_{y \in U} \mathbf{v}_y^+ \right)    
  \end{eqnarray*}
  Subtracting these, and using that $\lambda_1 < \lambda_1^+$ and $\mathbf{v}_{x_k} = \mathbf{v}_{x_k}^+ = 1$, we get
   \[ \mathbf{v}_{x_{k-1}}^+ - \mathbf{v}_{x_{k-1}} \leq \frac{1}{\lambda_1} \left( \mathbf{v}_{x_{k-2}}^+ - \mathbf{v}_{x_{k-2}} + \mathbf{v}_{z}^+ + \sum_{y \in U} \mathbf{v}_{y}^+ - \mathbf{v}_{y}\right) .\]
  By lemma~\ref{large_nbds}, we have $\sum_{y \in U} \mathbf{v}_{y}^+ - \mathbf{v}_y \leq 1$.  We also have
  $\mathbf{v}_{x_{k-2}}^+ - \mathbf{v}_{x_{k-2}} < 1$ and $\mathbf{v}_z^+ \leq 1$.  Hence
  $\mathbf{v}_{x_{k-1}}^+ - \mathbf{v}_{x_{k-1}} \leq 3 / \lambda_1$,
  or
   \[ \mathbf{v}_{x_{k-1}}^+ \geq \left( 1 + \frac{3}{\lambda_1 \mathbf{v}_{x_{k-1}}} \right) \mathbf{v}_{x_{k-1}}\]


  We can only apply lemma~\ref{change} if $\mathbf{v}_{x_{k}}^+$ is the
  largest eigenvector entry in $G_+$.  So we must consider two cases.


  \noindent \textbf{Case 1:} If in
  $G^+$ the largest eigenvector entry is still attained by vertex $\mathbf{v}_{x_k}$, then
  we can apply lemma~\ref{change}, and see that $\gamma(G^+) > \gamma(G)$
  if
   \[ \frac{\mathbf{v}_{x_{k-1}}}{2 \lambda_1^2} \geq \frac{12}{\lambda_1 \mathbf{v}_{x_{k-1}} n} \]
  or equivalently
   \[ \mathbf{v}_{x_{k-1}}^2 \geq \frac{24 \lambda_1}{n} .\]
  We have that $\lambda_1 = (1+o(1)) (n - n / \log(n))$, so it suffices for
  \begin{equation}\label{eqn_need}
   \mathbf{v}_{x_{k-1}} \geq \frac{5}{\sqrt{\log n}} .
  \end{equation}
  We know that
   \[ \mathbf{v}_{x_{k-1}} > \frac{|U| - 1}{2 \lambda_1} .\]
  By assumption
   \[ |U| + 2 = N(x_{k-1}) \geq 11 |C| / \sqrt{\log n} \]
  Equation~\ref{eqn_need} follows from this, so $\gamma(G^+) > \gamma(G)$.  

  \noindent \textbf{Case 2:} Say the largest eigenvector entry of $G^+$ is no
  longer attained by vertex $x_k$.  It is easy to see that the largest
  eigenvector entry is not attained by a vertex with degree less than or equal
  to $2$, and comparing the neighborhood of any vertex in $C$ with the neighborhood
  of $x_{k}$ we can see that $\mathbf{v}_{x_k} \geq \mathbf{v}_y$ for all $y \in C$.  So the
  largest eigenvector entry must be attained by $\mathbf{v}_{x_{k-1}}$.
  Then equation~\ref{pr_form2} no longer holds, instead we have
   \begin{equation}\label{case2} \gamma(G_+) = \frac{\sigma_+^{k-1} - \sigma_+^{-k+1}}{\sigma_+ - \sigma_+^{-1}} .\end{equation}
Recall that in lemma~\ref{change} we determined the change from $\gamma(G_+)$ to $\gamma(G)$ by considering $\lambda^+_1 - \lambda_1$ and $\mathbf{v}_{x_{k-1}}^+ - \mathbf{v}_{x_{k-1}}$. In this case, by \eqref{case2}, we must consider $\lambda^+_1 - \lambda_1$ and $1-\mathbf{v}_{x_{k-1}}$.
   Now if  $\mathbf{v}_{x_{k-1}}^+ > \mathbf{v}_{x_k}^+$ , then vertex $x_{k-1}$ in $G$
  is connected
  to all of $C$ except perhaps a single vertex.  Hence
  in $G$, the vertex $x_{k-1}$ is connected to all of $C$ except at most
  two vertices.  This gives the bound
   \[ 1 - \mathbf{v}_{x_{k-1}} \leq 3 / \lambda_1 \]
  and so as in the previous case, $\gamma(G_+) > \gamma(G)$.


  So in all cases, $x_{k-1}$ is connected to all vertices in $C$ that have
  eigenvector entry larger than $1/2$.  If all vertices in $C$ have eigenvector
  entry larger than $1/2$, then $x_{k-1}$ is connected to all of $C$,
  and this implies that $\mathbf{v}_{x_{k-1}} > \mathbf{v}_{x_k}$, which is a contradiction.
  At most one vertex in $C$ is smaller than $1/2$, and so
  there is a single vertex $z \in C$ with $\mathbf{v}_z < 1/2$.  We will
  quickly check that adding the edge $\{ x_{k-1}, z\}$ increases
  the principal ratio.  As before let $G_+$ be the graph obtained by
  adding this edge.  The largest eigenvector entry in $G_+$ is
  attained by $x_{k-1}$, as its neighborhood strictly contains the neighborhood
  of $x_{k}$.  
  As above, adding the edge $\{ z, x_{k}\}$ increases the spectral radius
  at least
   \[ \lambda_1^+ > \left(1 + \frac{\mathbf{v}_z}{2\lambda_1^2} \right) \lambda_1 \]
  and we have $1 - \mathbf{v}_{x_{k-1}} < 1 - \mathbf{v}_z/\lambda_1$.  Applying lemma~\ref{change}
  we see that $\gamma(G_+) > \gamma(G)$, which is a contradiction.
  Finally we conclude that the degree of $x_{k-1}$ must be smaller than
  $11 |C| / \sqrt{\log n}$.
\end{proof}

We note that this lemma gives that $\mathbf{v}_{x_{k-1}} < 1/2$ which implies that any vertex in $C$ has eigenvector entry larger than $1/2$.

\begin{lemma}\label{k_1_lemma}
 The vertex $x_{k-1}$ has degree exactly $2$ in $G$.  It follows that
 $\mathbf{v}_{x_{k-1}} < 2 / \lambda_1$.
\end{lemma}
\begin{proof}
  Let $U = N(x_{k-1}) \cap C$, $c = |U|$.  If $c=0$ then we are done.
  Otherwise let $G_-$ be the graph obtained from $G$ by deleting these
  $C$ edges.  We will show that $\gamma(G_-) > \gamma(G)$.

  \noindent \textbf{(1) Change in $\lambda_1$:}
  We have by equation~\ref{rayleigh quotient}, 
   \[ \lambda_1 - \lambda^{-}_1 \leq 2c \frac{\mathbf{v}_{x_{k-1}}}{||\mathbf{v}||_2^2}\]
  By Cauchy--Schwarz,
   \[ ||\mathbf{v}||_2^2 > \sum_{x \in N(x_{k})} \mathbf{v}_x^2 \geq \frac{\left(\sum_{x \in N(x_k)} \mathbf{v}_x\right)^2}{|C|+1} \geq \frac{(n-k)^2}{n-k+1}\]
  We also have
   \[ \mathbf{v}_{x_{k-1}} \leq \frac{c+2}{\lambda_1}\]
  Combining these we get
   \[ \lambda_1 - \lambda_1^{-} < \frac{9c^2}{\lambda_1 (n-k+1)} \Rightarrow \lambda_1 < \left(1 + \frac{9c^2}{\lambda_1 \lambda_1^{-} (n-k+1)}\right) \lambda_1^{-}\]
  We have $\lambda_1 \lambda_1^{-} > (n-k)^2$, so
  \[ \lambda_1 < \left( 1 + \frac{10c^2}{(n-k)^3} \right) \lambda_1^{-} \]
  
  \noindent \textbf{(2) Change in $\mathbf{v}_{x_{k-1}}$:}
  At this point, we know that in $G_-$ the vertices $x_1,\cdots , x_{k}$ form a pendant path, and so by the proof of lemma~\ref{path_bound}, we have $\mathbf{v}_{x_{k-1}}^- = (1+o(1)) / \lambda_1$. By the eigenvector equation and using that the vertices in $C$ have eigenvector entry at least $1/2$, we have $\mathbf{v}_{x_{k-1}} > (1 + c/2) / \lambda_1$.  So
   \begin{equation*}
     \mathbf{v}_{x_{k-1}} - \mathbf{v}_{x_{k-1}}^{-} > \frac{1}{\lambda_1} \left( \frac{c}{2} + o(1) \right)
   \end{equation*}
   In particular,
    \[ \mathbf{v}_{x_{k-1}} > \left( 1 + \frac{c}{3 \mathbf{v}_{x_{k-1}}^{-}\lambda_1}\right) \mathbf{v}_{x_{k-1}}^-\]
   Applying lemma~\ref{change}, it suffices now to show that
   \begin{equation}\label{eqn_need2}
    \frac{10c^2}{(n-k)^3} \exp \left(2 \frac{10c^2}{(n-k)^3} \lambda_1^- \log n \right) < \frac{c}{9 \mathbf{v}_{x_{k-1}}^- \lambda_1 n} .
   \end{equation}
   Now
    \[ \frac{10c^2}{(n-k)^3} < 10 \frac{11^2}{\log(n)} \frac{|C|^2}{(n-k)^3} < \frac{11^3}{\log n} \frac{\log n}{n} = \frac{11^3}{n} .\]
    Similarly $2 \frac{10c^2}{(n-k)^3} \lambda_1^- \log n < 2\cdot 11^3$, so the lefthand side of equation~\ref{eqn_need2} is smaller than $C_0 / n$, where
   $C_0$ is an absolute constant.
   For the righthand side, recall that $\mathbf{v}_{x_{k-1}}^- \lambda_1 = 1 + o(1)$, and also that
    \[ c > \frac{11}{\sqrt{\log n}} \left( \frac{n}{\log n} + o(1) \right) > \frac{10n}{\log^{3/2} n} .\]
   So the righthand side is larger than $1 / \log^{3/2}{n}$.  Hence for large
   enough $n$, the righthand side is larger than the lefthand side.
  
\end{proof}

We are now ready to prove the main theorem.

{
\renewcommand{\thetheorem}{1}
\begin{theorem}
  For sufficiently large $n$, the connected graph $G$ on $n$
  vertices with largest principal ratio is a kite graph.
\end{theorem}
\addtocounter{theorem}{-1}
}
\begin{proof}
  It remains to show that $C$ induces a clique. Assume it does not, and let $H$ be the graph $P_k \cdot K_{n-k+1}$. We will show
  that $\gamma(H) > \gamma(G)$, and this contradiction tells us that $C$ is a clique. As before, lemma \ref{path_bound} gives that
\[
\gamma(H)  = \frac{\sigma_H^k - \sigma_H^{-k}}{\sigma_H - \sigma_H^{-1}},
\]
where
\[
\sigma(H) = \frac{\lambda_1(H) - \sqrt{\lambda_1(H)^2 -4}}{2}.
\]

Since $x_1,\cdots x_k$ form a pendant path we also know that 
\[
\gamma(G) = \frac{\sigma^k - \sigma^{-k}}{\sigma - \sigma^{-1}}.
\]

Now, $\lambda_1(H) > \lambda_1(G)$ because $E(G) \subsetneq E(H)$. Since the functions $g(x) = x+\sqrt{x^2-4}$ and $f(x) = (x^k - x^{-k})/(x-x^{-1})$ are increasing when $x\geq 1$, we have $\gamma(H) > \gamma(G)$.

 \end{proof}



\section{Connected Graphs of Maximum Irregularity}\label{pineapple}

\subsection{Structural Lemmas}
Throughout this section, let $G$ be a graph on $n$ vertices with spectral radius $\lambda_1$ and first eigenvector normalized so that $x=1$. Throughout we will use $d = 2e(G)/n$ to denote the average degree. We will also assume that $G$ is the connected graph on $n$ vertices that maximizes $\lambda_1 - d$.

To show that $G$ is a pineapple graph we first show that $\lambda_1 \sim \frac{n}{2}$ and $d\sim \frac{n}{4}$ (Lemma \ref{spectral radius and average degree}). Then we show that there exists a vertex with degree close to $\frac{n}{2}$ and eigenvector entry close to $1$ (Lemma~\ref{u good}). We bootstrap this to show that there are many vertices of degree about $\frac{n}{2}$, that these vertices induce a clique, and further that most of the remaining vertices have degree $1$ (Lemma \ref{modifying conditions} and Proposition \ref{almost pineapple structure}).  We complete the proof by showing that all vertices not in the clique have degree $1$ and that they are all adjacent to the same vertex.

We remark that once we show that $G$ is a pineapple graph, the small question remains of {\em which} pineapple graph maximizes $\lambda_1 - d$. Optimization of a cubic polynomial shows that $G$ is a pineapple with clique size $\lceil \frac{n}{2}\rceil +1$ (see \cite{AouchicheEtAl2008}, section 6).

\begin{figure}[]
\begin{center}
\begingroup

\setlength{\unitlength}{.01cm}
{
\setlength{\fboxsep}{10pt}
\framebox[1.5\width]{
\begin{tikzpicture}[rotate=180] 
   
   \coordinate (c1) at ($(7,0) + (180:1)$);
   \coordinate (c2) at ($(7,0) + (225:1)$);
   \coordinate (c3) at ($(7,0) + (270:1)$);
   \coordinate (c4) at ($(7,0) + (315:1)$);
   \coordinate (c5) at ($(7,0) + (0:1)$);
   \coordinate (c6) at ($(7,0) + (45:1)$);
   \coordinate (c7) at ($(7,0) + (90:1)$);
   \coordinate (c8) at ($(7,0) + (135:1)$);
   
   \coordinate (e1) at ($(7,0) + (270:1) + (0.9,-1.0)$);
   \coordinate (e2) at ($(7,0) + (270:1) + (0.5,-1.0)$);
   \coordinate (e3) at ($(7,0) + (270:1) + (-0.5,-1.0)$);
   \coordinate (e4) at ($(7,0) + (270:1) + (-0.9,-1.0)$);
   
   \coordinate (b1) at ($(7,0) + (270:1) + (0.15,-1.0)$);
   \coordinate (b2) at ($(7,0) + (270:1) + (0,-1.0)$);
   \coordinate (b3) at ($(7,0) + (270:1) + (-0.15,-1.0)$);
   
   \coordinate (lb2) at ($(7,0) + (270:1) + (0,-1.4)$);   
      
  
%   \draw[draw=black] (c1) -- (c2);
%   \draw[draw=black] (c1) -- (c3);
%   \draw[draw=black] (c1) -- (c4);
%   \draw[draw=black] (c1) -- (c5);
%   \draw[draw=black] (c1) -- (c6);
%   \draw[draw=black] (c1) -- (c7);  
%   \draw[draw=black] (c1) -- (c8);
%
%   \draw[draw=black] (c2) -- (c3);
%   \draw[draw=black] (c2) -- (c4);
%   \draw[draw=black] (c2) -- (c5);
%   \draw[draw=black] (c2) -- (c6);
%   \draw[draw=black] (c2) -- (c7);  
%   \draw[draw=black] (c2) -- (c8); 
%
%   \draw[draw=black] (c3) -- (c4);
%   \draw[draw=black] (c3) -- (c5);
%   \draw[draw=black] (c3) -- (c6);
%  \draw[draw=black] (c3) -- (c7);  
%   \draw[draw=black] (c3) -- (c8); 
%
%   \draw[draw=black] (c4) -- (c5);
%   \draw[draw=black] (c4) -- (c6);
%   \draw[draw=black] (c4) -- (c7);  
%   \draw[draw=black] (c4) -- (c8);
%   
%   \draw[draw=black] (c5) -- (c6);
%   \draw[draw=black] (c5) -- (c7);  
%   \draw[draw=black] (c5) -- (c8); 
%
%   \draw[draw=black] (c6) -- (c7);  
%   \draw[draw=black] (c6) -- (c8);  
% 
%   \draw[draw=black] (c7) -- (c8);
    
   \draw[draw=black] (c3) -- (e1);   
   \draw[draw=black] (c3) -- (e2);   
   \draw[draw=black] (c3) -- (e3);   
   \draw[draw=black] (c3) -- (e4);   
   
   %\draw (7,0) circle [radius=1];   
   \draw[black] (7,0) node[circle,minimum size=2cm,draw] (v1) {$K_m$};

   %\filldraw[fill=blue,draw=blue] (c1) circle [radius=0.07];
   %\filldraw[fill=blue,draw=blue] (c2) circle [radius=0.07];
   \filldraw[fill=blue,draw=blue] (c3) circle [radius=0.07];
   %\filldraw[fill=blue,draw=blue] (c4) circle [radius=0.07];
   %\filldraw[fill=blue,draw=blue] (c5) circle [radius=0.07];
   %\filldraw[fill=blue,draw=blue] (c6) circle [radius=0.07];
   %\filldraw[fill=blue,draw=blue] (c7) circle [radius=0.07];
   %\filldraw[fill=blue,draw=blue] (c8) circle [radius=0.07];

   \filldraw[fill=blue,draw=blue] (e1) circle [radius=0.07];
   \filldraw[fill=blue,draw=blue] (e2) circle [radius=0.07];
   \filldraw[fill=blue,draw=blue] (e3) circle [radius=0.07];
   \filldraw[fill=blue,draw=blue] (e4) circle [radius=0.07];
   
   \filldraw[black] (lb2) node {$n$};   
   
   \filldraw[fill=black] (b1) circle [radius=0.03];
   \filldraw[fill=black] (b2) circle [radius=0.03];   
   \filldraw[fill=black] (b3) circle [radius=0.03];
      
\end{tikzpicture}}
}
\endgroup
\end{center}
\caption{The pineapple graph, $PA(m,n)$.}
   \label{fig-pn4}
\end{figure}


\begin{lemma}\label{spectral radius and average degree}
We have $\lambda_1(G) = \frac{n}{2} + c_1\sqrt{n}$ and $\frac{2e(G)}{n} = \frac{n}{4} + c_2\sqrt{n}$, where $|c_1|, |c_2|<1$.
\end{lemma}

\begin{proof}
By eigenvalue interlacing, $\mathrm{PA}(p,q)$ has spectral radius at least $p-1$. Setting 
 $H = \mathrm{PA}\left( \left\lceil \frac{n}{2}\right\rceil +1, \left\lfloor\frac{n}{2}\right\rfloor-1\right)$, we have 
\[
\lambda_1(H) - \frac{2e(H)}{n} \geq \frac{n}{4} - \frac{3}{2}.
\]
On the other hand, an inequality of Hong \cite{Hong1988} gives 
\[
\lambda_1^2 \leq 2e(G) - (n-1).
\]
It follows that
\begin{equation}\label{lower bound on average degree}
d \geq \frac{\lambda_1^2}{n} + 1 - \frac{1}{n}.
\end{equation}
Setting $\lambda_1 = pn$ and applying \eqref{lower bound on average degree}, we have $\lambda_1 - d \leq pn - p^2n -1 + \frac{1}{n}$. The right hand side of the inequality is maximized at $p=1/2$, giving 
\begin{equation}\label{tight bound on irregularity}
\frac{n}{4} - \frac{3}{2} \leq \lambda_1 - d \leq \frac{n}{4} - 1 + \frac{1}{n}.
\end{equation}

Next setting $\lambda_1 = \frac{n}{2} + c_1 \sqrt{n}$, \eqref{lower bound on average degree} gives
\[
d \geq \frac{n}{4} + c_1\sqrt{n} + c_1^2 + 1 - \frac{1}{n},
\]
whereas \eqref{tight bound on irregularity} implies
\begin{equation}\label{d bar bound}
d \leq \lambda_1 - \frac{n}{4} + \frac{3}{2}  = \frac{n}{4} + c_1\sqrt{n}  + \frac{3}{2}.
\end{equation}
Together, these imply $|c_1| <1$ and prove both statements for $n$ large enough.
\end{proof}

\begin{lemma}\label{error in x neighborhood small}
There exists a constant $c_3$ not depending on $n$ such that 
\[
0 \leq \frac{1}{|N(x)|} \sum_{y\sim x} d_y - \lambda_1 \mathbf{v}_y \leq c_3\sqrt{n}.
\]
\end{lemma}

\begin{proof}
From the inequality of Hong,
\[
\sum_{y\sim x} \lambda_1 \mathbf{v}_y = \lambda_1^2 \leq dn - (n-1).
\]
Rearranging and applying Lemma \ref{spectral radius and average degree}, we have
\[
0\leq \sum_{y\sim x} \left( d_y - \lambda_1 \mathbf{v}_y \right) = O\left(n^{3/2}\right).
\]
By equation~\eqref{eigenvector equation} again, and because the first eigenvector is normalized with $\mathbf{v}_x=1$, we have
\[
\lambda_1 = \sum_{y\sim x} \mathbf{v}_y \leq d_x,
\]
giving $d_x = \Omega(n)$. Combining, we have 
\[
\frac{1}{|N(x)|} \sum_{y\sim x} \left( d_y - \lambda_1 \mathbf{v}_y \right) = O\left(\sqrt{n}\right),
\]
where the implied constant is independent of $n$.
\end{proof}

Now we fix a constant $\epsilon > 0$, whose exact value will be 
chosen later.  The next lemma implies that close to half of the vertices of
$G$ have eigenvector entry close to $1$ for $n$ sufficiently large, 
depending on the chosen $\epsilon$.  We follow that with a proposition
which outlines the approximate structure of $G$, and then finally use
variational arguments to deduce that $G$ is exactly a pineapple graph.


\begin{lemma}\label{u good}
 There exists a vertex $u\not=x$ with $\mathbf{v}_u > 1- 2 \epsilon$ and $d_u - \lambda_1 \mathbf{v}_u = O(\sqrt{n})$.  Moreover $d_u \geq \left( 1/2 - 2\epsilon \right)n$.
\end{lemma}


%\begin{lemma}\label{y good}
%Let $\epsilon >0$. There exists a $y\in N(x)$ such that $y> \frac{1}{2} - \epsilon$ and $d_y - \lambda_1 y < c_\epsilon \sqrt{n}$ where $c_\epsilon$ is a constant depending only on $\epsilon$.
%\end{lemma}

\begin{proof}
We proceed by first showing a weaker result: that there is a vertex $y$ with
$\mathbf{v}_y> \frac{1}{2} - \epsilon$ and $d_y - \lambda_1 \mathbf{v}_y = O(\sqrt{n})$, and 
additionally that $y \in N(x)$.  We will then use this to obtain
the required result.


Let $A: = \{z\sim x: \mathbf{v}_z> \frac{1}{2} - \epsilon\}$. By Lemma \ref{spectral radius and average degree},
\[
\lambda_1 = \frac{n}{2} + c_1\sqrt{n},
\]
where $|c_1| <1$. Since $0< \mathbf{v}_z\leq 1$ for all $z\sim x$, we see that $|A| \geq \delta_\epsilon n$ where $\delta_\epsilon$ is a positive constant that depends only on $\epsilon$. Let $B = \{z\sim x: d_z - \lambda_1 \mathbf{v}_z >  K\sqrt{n}\}$, where $K$ is a fixed constant whose exact value will be chosen later. Now
\[
\frac{1}{|N(x)|} \sum_{y\sim x} \left( d_y -\lambda_1 \mathbf{v}_y \right) \geq \frac{1}{|N(x)|} \sum_{z\in B} \left( d_z - \lambda_1 \mathbf{v}_z \right) \geq \frac{1}{n} |B| K\sqrt{n}.
\]
By Lemma \ref{error in x neighborhood small}, $|B| \leq \frac{c_3}{K} n$. Therefore, for $K$ large enough depending only on $\epsilon$, we have $\left|A\cap B^c\right| > 0$.  This proves the existence of the vertex $y$,
with the properties claimed at the beginning of the proof.


Next, we show that there exists a set $U\subset N(y)$ such that 
$|U| \geq \left(\frac{1}{4} - 2\epsilon \right)n$ and 
$\mathbf{v}_u \geq 1-2\epsilon$ for all $u\in U$.  By Lemma \ref{spectral radius and average degree},
\[
\left(\frac{n}{2} + c_1\sqrt{n}\right)\left(\frac{1}{2} - \epsilon\right) \leq \lambda_1 \mathbf{v}_y \leq d_y,
\]
where $|c_1| < 1$. So $d_y \geq \left(\frac{1}{4} - \epsilon \right) n$ for $n$ large enough. Now let $C = \{z\sim y: \mathbf{v}_z < 1-2\epsilon \}$. Then
\[
K \sqrt{n} \geq d_y - \lambda_1 \mathbf{v}_y = \sum_{z\sim y} \left( 1 - \mathbf{v}_z \right) \geq \sum_{z\in C} \left( 1-\mathbf{v}_z \right) \geq 2|C| \epsilon.
\]
Therefore
\[
|N(y) \setminus C| \geq \left(\frac{1}{4} - \epsilon\right)n  - \frac{K \sqrt{n}}{2\epsilon}.
\]
Setting $U = N(y) \setminus C$, we have $|U| > \left(\frac{1}{4} - 2\epsilon\right)n$ for $n$ large enough. 


Set $D = U \cap N(x)$.  We will first find a lower bound on $|D|$.  We have
 \begin{equation*}
  \lambda_1^2 \leq \sum_{y \sim x} d_y \leq 2m - \sum_{y \not \in N(x)} d_y.
 \end{equation*}
Rearranging this we get 
 \[ d - \frac{\lambda_1^2}{n} \geq \frac{1}{n} \sum_{y \not \in N(x)} d_y.\]
Now applying the bound on $d$ from equation~\ref{d bar bound} and expression for $\lambda_1$ in Lemma~\ref{spectral radius and average degree} yields
 \[ \left( \frac{n}{4} + c_1 \sqrt{n} + \frac{3}{2}\right) - \frac{\left(\frac{n}{2} + c_1 \sqrt{n}\right)^2}{n} \geq \frac{1}{n} \sum_{y \not \in N(x)} d_y, \]
which implies that
 \[ \frac{3}{2} n \geq \left( \frac{3}{2} - c_1^2 \right) n \geq \sum_{y \not \in N(x)} d_y \geq \sum_{y \in U\setminus N(x)} d_y \geq |U\setminus N(x)| (1-2\epsilon) \lambda_1 .\]
So 
 \[ |U \setminus N(x)| \leq \frac{3}{2(1-2\epsilon)} \frac{n}{\lambda_1} = \frac{3}{2(1-2\epsilon)} \frac{1}{1/2 + c_1 n^{-1/2}} .\]
In particular, $|D| \geq (\frac{1}{4} - c'_\epsilon) n$.

Now by the same argument used at the start of the proof to show the existence
of the vertex $y$, we have some vertex $u \in D$
with $d_u - \lambda_1 \mathbf{v}_u = O(\sqrt{n}$).   Finally 
\[d_u \geq \mathbf{v}_u \lambda_1 \geq (1 - 2 \epsilon) (n/2 + c_1\sqrt{n}) \geq \left( 1/2 - 2\epsilon \right)n .  \]

\end{proof}

\subsection{Alteration Step}

\begin{lemma}\label{modifying conditions}
 Let $x,y$ be two vertices in $G$.  If
 $\mathbf{v}_x\mathbf{v}_y > 1/2 +n^{-1/2} + 5n^{-1}$, then $x$ and $y$ are adjacent.  On the
 other hand, if $\mathbf{v}_x\mathbf{v}_y < 1/2 - 3\epsilon$ then $x$ and $y$ are not adjacent.
\end{lemma}
\begin{proof}
We begin by bounding the dot product of the leading eigenvector $\textbf{v}$
with itself.  We will show that
\begin{equation}\label{vvt bound}
\frac{n}{2} + \sqrt{n} + 5 \geq \textbf{v}^t\textbf{v} > \frac{n}{2} - 2 \epsilon n  - O(\sqrt{n}).
\end{equation}


\noindent First, we show the lower bound.  With $u$ from the previous lemma, by Cauchy--Schwarz we have
\[  \textbf{v}^t\textbf{v} \geq \sum_{z \sim u} \mathbf{v}_z^2 \geq \frac{1}{d_u}\left( \sum_{z \sim u} \mathbf{v}_z \right)^2 = \frac{(\lambda_1 \mathbf{v}_u)^2}{d_u}. \]
By Lemma~\ref{u good}, we then have
 \[\textbf{v}^t\textbf{v} \geq \frac{(d_u - O(\sqrt{n}))^2}{d_u} \geq d_u - O(\sqrt{n}) > \frac{n}{2} - 2 \epsilon n  - O(\sqrt{n}) .\]
For the upper bound of inequality~\eqref{vvt bound}, first set $E = \left( N(x) \cup \{x\}\right)^C$.  Then
 \[ \textbf{v}^t\textbf{v} = \sum_{z \in V(G)} \mathbf{v}_z^2 \leq \sum_{z \in V(G)} \mathbf{v}_z \leq 1 + \sum_{z \in N(x)} \mathbf{v}_z + \sum_{z \in E} \mathbf{v}_z \leq 1 + \lambda_1 + \frac{1}{\lambda_1} \sum_{z \in E} d_z .\]
From the proof of Lemma~\ref{u good} we have the bound
 \[ \sum_{ z \in E} d_z \leq \frac{3}{2} n .\]
Hence
 \[ \textbf{v}^t \textbf{v}  \leq 1 + \frac{n}{2} + c_1 \sqrt{n} + \frac{3}{2} \cdot \frac{1}{1/2 + c_1 n^{-1/2}} \leq \frac{n}{2} + \sqrt{n} + 5 .\]
This completes the proof of inequality~\eqref{vvt bound}.


 Let $\lambda_1^+$ be the leading eigenvalue of the graph formed by adding the 
 edge $\{x,y\}$ to $G$.  Then by \eqref{rayleigh quotient} we have
  \[ \lambda_1^+ - \lambda_1 \geq \frac{\textbf{v}^t (A^+-A) \textbf{v}}{\textbf{v}^t \textbf{v}} \geq \frac{2\mathbf{v}_x\mathbf{v}_y}{\textbf{v}^t \textbf{v}} \geq \frac{2\mathbf{v}_x\mathbf{v}_y}{n/2 + \sqrt{n} + 5}  = \frac{2\mathbf{v}_x\mathbf{v}_y}{n(1/2+n^{-1/2} + 5 n^{-1})}.\]
If $\mathbf{v}_x\mathbf{v}_y > 1/2+n^{-1/2} + 5n^{-1}$, then
\[ (\lambda_1^+ - d^+) - (\lambda - d) > \frac{2}{n} - \frac{2}{n} = 0.\]
Hence $\left\{x,y\right\}$ must already have been an edge, otherwise this
would contradict the maximality of $G$.

Similarly if $\lambda_1^-$ is the leading eigenvalue of the graph obtained
from $G$ by deleting the edge $\{x,y\}$, then  
  \[ \lambda_1 - \lambda_1^- \leq \frac{\textbf{v}^t (A-A^-) \textbf{v}}{\textbf{v}^t \textbf{v}} \leq \frac{2\mathbf{v}_x\mathbf{v}_y}{n/2 - 2\epsilon n - O(\sqrt{n})} \leq \frac{2\mathbf{v}_x\mathbf{v}_y}{(1/2 - 3 \epsilon)n},\]
when $n$ is large enough.  Now if
$\mathbf{v}_x\mathbf{v}_y < 1/2 - 3\epsilon$, then 
 \[ (\lambda_1 - d) - (\lambda_1^- - d^-) < 0 .\] %XXX: this doesn't typeset well

\end{proof}

%%%%%%%%%%%%%%%%%%picture%%%%%%%%%%%%%

\begin{figure}[]
\begin{center}
\begingroup

\setlength{\unitlength}{.01cm}
{
\setlength{\fboxsep}{10pt}
\framebox[1.5\width]{
\begin{tikzpicture}
\filldraw[black] (0.553380,1.150606) node[label={[label distance=0.1cm]180:$1 - O(\epsilon)$},label={[label distance=0.1cm]$U$},circle,minimum size=2cm,draw] (v0) {$K_{|U|}$};

\filldraw[black] (4.553380,1.150606) node[label={[label distance=0.1cm]0:$O(\frac{\epsilon}{n})$},label={[label distance=0.1cm]$V$},circle,minimum size=2cm,draw] (v1) {$I_{|V|}$};

\filldraw[black] (2.553380,-1.80606) node[label={[label distance=0.1cm]0:$\frac{1}{2}+O(\epsilon)$},label={[label distance=0.1cm]88:$W$},circle,minimum size=0.8cm,draw] (v1) {$I_{|W|}$};

\coordinate (UC) at (0.553380,1.150606);
\coordinate (VC) at (4.553380,1.150606);
\coordinate (WC) at (2.553380,-1.80606);

\coordinate (Ul1) at ($(UC) + (1.1,-0.4)$);
\coordinate (Ul2) at ($(UC) + (1.18,-0)$);
\coordinate (Ul3) at ($(UC) + (1.1,0.4)$);

\coordinate (Ub2) at ($(UC) + (0.24,-1.2)$);
\coordinate (Ub3) at ($(UC) + (0.54,-1.0)$);


\coordinate (Vl1) at ($(VC) + (-1.1,-0.4)$);
\coordinate (Vl2a) at ($(VC) + (-1.18,-0.2)$);
\coordinate (Vl2b) at ($(VC) + (-1.18,0.2)$);
\coordinate (Vl3) at ($(VC) + (-1.1,0.4)$);

\coordinate (Wt1) at ($(WC) + (-0.5,0.5)$);
\coordinate (Wt2) at ($(WC) + (-0.3,0.6)$);
\coordinate (Wt3) at ($(WC) + (-0.6,0.3)$);


\draw (Ul1) -- (Vl1);
\draw (Ul2) -- (Vl2a);
\draw (Ul2) -- (Vl2b);
\draw (Ul3) -- (Vl3);

\draw (Ub3) -- (Wt1);
\draw (Ub3) -- (Wt2);
\draw (Ub3) -- (Wt3);

\draw (Ub2) -- (Wt3);
\draw (Ub2) -- (Wt1);

%\draw[black] (v0) -- (v1);

\end{tikzpicture}}
}
\endgroup
\end{center}
\caption[Structure of $G$ in Proposition \ref{almost pineapple structure}.]{
  Structure of $G$ in Proposition \ref{almost pineapple structure}.  The number beside
  each set indicates the values of eigenvector entries in the set.
  $U$ is a clique and $V$, $W$ are independent sets.  Each vertex in $V$ is adjacent
  to exactly one vertex in $U$, and each vertex in $W$ is adjacent to multiple
  vertices in $U$.
   \label{pineapple structure picture}}
\end{figure}

%%%%%%%%%%%%%%%%%%%%%%%%

\begin{proposition}\label{almost pineapple structure}
For $n$ sufficiently large, we can partition the vertices of $G$ into three 
sets $U, V, W$ (see Figure \ref{pineapple structure picture}) where 
\begin{itemize}
 \item[(i)] vertices in $V$ have eigenvector entry smaller than $(2+\epsilon) / n$ and have degree one, 
 \item[(ii)] vertices in $U$ induce a clique, 
 all have eigenvector entry larger than $1 - 20\epsilon$, and $(1/2 - 3\epsilon) n \leq |U| \leq (1/2 + \epsilon)n$,

 \item[(iii)] vertices in $W$ have eigenvector entry in the range $\left[1/2 - 4\epsilon,  1/2 + 21 \epsilon \right]$ and are adjacent only to vertices in $U$.  
\end{itemize}
\end{proposition}

\begin{proof}
  By Lemma~\ref{modifying conditions}, any two vertices in $G$ with eigenvector entry $1$ are adjacent.  Moreover, it is easy to see that
  every vertex in $G$ is incident to at least one vertex with eigenvector entry $1$:  if not, for each vertex not incident to a vertex
  with eigenvector entry $1$, delete one of its edges and add a new edge from that vertex to a vertex with eigenvector entry $1$ (such as
  the vertex $x$).  The resulting graph is connected, will have the same number of edges as the original graph, and will have strictly larger $\lambda_1$
  (this can be seen by considering the Rayleigh quotient, as in the proof of Lemma~\ref{modifying conditions}).
  So by maximality of $G$, there are no such vertices.  This implies that the set of edges that are incident
  to a vertex with eigenvector entry $1$ spans the vertex set of $G$. 
  In particular, if we remove any edge that is not incident to a vertex with eigenvector entry $1$, we do not disconnect the graph.  We will use this fact repeatedly in this proof.

 
  
\begin{itemize}
\item[(i)] Let $V$ consist of all vertices in $G$ with eigenvector entry 
less than $1/2 - 4 \epsilon$.  By Lemma~\ref{modifying conditions}, removing 
any edge incident to a vertex in $V$ strictly increases $\lambda_1 - d$, so each vertex in $V$ has degree one.  By equation~\eqref{eigenvector equation},
the eigenvector entry of any such vertex is at most $1/\lambda_1 < (2+\epsilon) / n $, when $n$ is large enough.
\item[(ii)] From Lemma~\ref{u good}, we have a vertex $u$ such that $d_u - \lambda_1 \mathbf{v}_u = O(\sqrt{n})$.  Let $X$ be the set of neighbors $z$ of $u$ such that $\mathbf{v}_z < 9/10$.  Then we have
 \[ (1 - 9/10)|X| \leq \sum_{y \sim u} 1 - \mathbf{v}_y = d_u - \lambda_1 \mathbf{v}_u = O(\sqrt{n}). \]
Hence $|X| = O(\sqrt{n})$.  Let $U$ be all vertices in $G$ with eigenvector entry at least $9/10$.  So, by Lemma~\ref{u good}
 \[ |U| \geq d_u - |X| \geq n/2 - 2 \epsilon n - O(\sqrt{n}) . \]
For $n$ large enough, we have $|U| \geq (1/2 - 3\epsilon) n$. For sufficiently large $n$, by Lemma~\ref{modifying conditions} these vertices are all adjacent to each other.  For the upper bound on $|U|$ we use the expression for $e(G)$ in Lemma~\ref{spectral radius and average degree}
\[ |U| (|U| - 1) \leq 2e(G) \leq  \frac{n^2}{4} + c_2n\sqrt{n}, \]
which implies $|U| \leq (1/2+\epsilon) n$ for large enough $n$.

Now take any vertex $y \in U$.  If $x$ is a vertex with largest eigenvector entry, then 
\begin{equation}\label{y_bound}
 \lambda_1 - \lambda_1 \mathbf{v}_y \leq \sum_{z \in N(x) \setminus N(y)} \mathbf{v}_z \leq \mathbf{v}_y + \sum_{z \in U^C} \mathbf{v}_z . 
\end{equation}
We have
 \begin{eqnarray*}
  \lambda_1 \sum_{z \in U^C} \mathbf{v}_z \leq \sum_{z \in U^C} d_z &\leq& 2e(G) - 2|E(U,U)|\\
   &\leq& \frac{n^2}{4} + c_2 n \sqrt{n} - (1/2 - 3\epsilon)(1/2 - 3 \epsilon - 1/n) n^2 \\
   &\leq & 4 \epsilon n^2 ,
 \end{eqnarray*}
for $n$ sufficiently large, where we are using the expression for $e(G)$ given by Lemma~\ref{spectral radius and average degree}.
In particular,
 \[ \sum_{z \in U^C} \mathbf{v}_z \leq 9 \epsilon n .\] 
Finally, by equation~\ref{y_bound} we have 
 \[ \mathbf{v}_y \geq 1 - \frac{1}{\lambda_1} \sum_{z \in U^C} \mathbf{v}_z -\frac{\mathbf{v}_y}{\lambda_1} \geq (1 - 20 \epsilon) .\]
 
\item[(iii)] Let $W$ consist of all remaining vertices of $G$.  If a vertex 
has eigenvector entry smaller than $1/2 - 4\epsilon$ then it is in $V$ by
construction.  If a vertex $z \in W$ has eigenvector entry larger than $1/2 + 21\epsilon$
then we have
 \[ (1/2 + 21 \epsilon) (1 - 20 \epsilon) > 1/2 + \epsilon, \]
if $\epsilon < 1/50$, say.   So for sufficiently large $n$, by Lemma~\ref{modifying conditions} we 
have that $z$ is adjacent to every vertex in $U$.  But by the proof of part (ii), this implies that $\mathbf{v}_z > 1 - 20\epsilon$, which contradicts $z \in W$.

For $z \in W$ and any vertex $y \in U^C$, then 
 $\mathbf{v}_y\mathbf{v}_z \leq (1/2 + 21 \epsilon)(1/2 + 21\epsilon) < 1/4 + 22 \epsilon$
and so by Lemma~\ref{modifying conditions} there is no edge between $y$ and 
$z$ in the maximal graph $G$.
\end{itemize}
\end{proof}

\subsection{The Pineapple Graph is Extremal}

\begin{theorem}
For sufficiently large $n$, $G$ is a pineapple graph.
\end{theorem}
\begin{proof}
  Take $U,V,W$ as in the previous lemma.  We begin by showing that the set $W$ must be empty.
  Proceeding by contradiction, let $z$ be in $W$.  Furthermore
let $G^+$ be the graph obtained by adding edges from $z$ to every vertex in $U$.
We will show that $\lambda_1(G^+) - d(G^+) > \lambda_1(G) - d(G)$, which contradicts the maximality of $G$.

Since the vertex $z$ is adjacent only to vertices in $U$, and the
fact that vertices in $U$ have eigenvector entry between $1-20\epsilon$ and $1$,
equation~\eqref{eigenvector equation} yields
 \[\lambda_1 (1/2 - 4 \epsilon) \leq \lambda_1 \mathbf{v}_z \leq d_z(G) \leq \frac{\lambda_1 \mathbf{v}_z}{1 - 20 \epsilon} = (1/2 + O(\epsilon))\lambda_1 .\]

\noindent Using the expression for $\lambda_1$ in Lemma~\ref{spectral radius and average degree}, for large enough $n$ we have
 \[ \left(1-\epsilon\right)\frac{n}{4} \leq d_z(G) \leq \left(1+\epsilon\right) \frac{n}{4} .\]
So we can bound the change in the average degrees
 \[ d(G^+) - d(G) \leq \frac{2(|U| - (1-\epsilon)n/4)}{n}< 1/2 + 3\epsilon .\]
 Next we find a lower bound on $\lambda_1(G^+) - \lambda_1(G)$.
 Let $\textbf{w}$ be the vector that is equal to $\textbf{v}$ on all 
vertices except $z$, and equal to $1$ for $z$.  Then, 
 \[ \lambda_1(G^+) \geq \frac{\textbf{w}^tA^+\textbf{w}}{\textbf{w}^t\textbf{w}} .\] 
We first find a lower bound for the numerator (with abuse of big-O notation with inequalities)
\begin{eqnarray*} 
\textbf{w}^tA^+\textbf{w} & \geq & \textbf{w}^tA\textbf{w} + 2(|U| - d_z(G))(1-O(\epsilon)) \geq \textbf{w}^tA\textbf{w} + (1/2-O(\epsilon))n \\
& \geq &  \textbf{v}^tA\textbf{v} + 2d_z(G) \left(1-\mathbf{v}_z\right)(1-20\epsilon) + (1/2-O(\epsilon))n \\
%& \geq &  \textbf{v}^tA\textbf{v} + 2d_z(G) \left(1/2 - 21\epsilon\right)(1-20\epsilon) + (1/2-O(\epsilon))n \\
& \geq &  \textbf{v}^tA\textbf{v} + 2d_z(G) \left(1/2 - 31\epsilon\right) + (1/2-O(\epsilon))n \\
%& \geq & \textbf{v}^tA\textbf{v} + 2(1-\epsilon)(n/4) \left(1/2 - 31\epsilon\right) + (1/2-O(\epsilon))n \\
%& \geq & \textbf{v}^tA\textbf{v} + (n/4) \left(1 - 63\epsilon\right) + (1/2-O(\epsilon))n\\
& \geq &  \textbf{v}^tA\textbf{v} + (3/4-O(\epsilon))n .
\end{eqnarray*}
Similarly, we find an upper bound for the denominator
\begin{eqnarray*}
\textbf{w}^t \textbf{w} &=& \textbf{v}^t\textbf{v} + 1 - \mathbf{v}_z^2 \\
&\leq& \textbf{v}^t\textbf{v} + 1 - (1/2 - 4\epsilon)^2 \\
&\leq& \textbf{v}^t\textbf{v} + 3/4 + 4\epsilon .
\end{eqnarray*}
Combining these, and using the bound on $\textbf{v}^t\textbf{v}$ from
the proof of Lemma~\ref{modifying conditions}, we get
\begin{eqnarray*}
 \lambda_1(G^+) - \lambda_1(G) &\geq& \frac{\textbf{w}^tA^+\textbf{w}}{\textbf{w}^t\textbf{w}} - \frac{\textbf{v}^tA\textbf{v}}{\textbf{v}^t\textbf{v}}\\
% & \geq & \frac{\textbf{v}^tA\textbf{v} + (1-37\epsilon)3n/4}{\textbf{v}^t\textbf{v} + 3/4 + 4\epsilon} -  \frac{\textbf{v}^tA\textbf{v}}{\textbf{v}^t\textbf{v}}\\
 &\geq& \frac{\textbf{v}^t\textbf{v}(3/4-O(\epsilon))n - \textbf{v}^tA\textbf{v} (3/4 + 4\epsilon)}{\textbf{v}^t\textbf{v} (\textbf{v}^t\textbf{v} + 3/4 + 4\epsilon)}\\
 &\geq & \frac{(3/4-O(\epsilon))n - (3/4 + 4\epsilon)\lambda_1(G)}{\textbf{v}^t\textbf{v} + 3/4 + 4\epsilon} \\ 
 &= &  3/4 + O(\epsilon) .
\end{eqnarray*}
Hence $\lambda_1(G^+) - \lambda_1(G) > d(G^+) - d(G)$, and by maximality of $G$ we conclude that $W = \emptyset$.

At this point we know that $G$ consists of a clique together with a set of pendant vertices $V$.  All that remains is to show that all of the pendant vertices  are incident to the same vertex in the clique.  Let $V = \left\{v_1, v_2, \cdots, v_k\right\}$, and let $u_i$ be the unique vertex in $U$ that $v_i$ is adjacent to.  Let $G^+$ be the graph obtained from $G$ by deleting the edges $\left\{v_i,u_i\right\}$ and adding the edges $\left\{v_i,x\right\}$, where $x$ is a vertex with eigenvector entry $1$.  Now, $d(G^+) = d(G)$, and 
\[ \lambda_1(G^+) -\lambda_1(G) \geq \frac{\textbf{v}^t A^+ \textbf{v}}{\textbf{v}^t\textbf{v}} - \frac{\textbf{v}^t A \textbf{v}}{\textbf{v}^t\textbf{v}}, \]
with equality if and only if $\textbf{v}$ is a leading eigenvector for $A^+$.  We have
\[\frac{\textbf{v}^t A^+ \textbf{v}}{\textbf{v}^t\textbf{v}} - \frac{\textbf{v}^t A \textbf{v}}{\textbf{v}^t\textbf{v}} = \frac{1}{\textbf{v}^t\textbf{v}}\left( \sum_{i=1}^k 1 - \mathbf{v}_{u_i} \right) \geq 0 ,\]
with equality if and only if $\mathbf{v}_{u_i} = 1$ for all $1 \leq i \leq k$.  By maximality of $G$, we have equality in both of the above inequalities, and so $\textbf{v}$ is a leading eigenvector for $G^+$, and every vertex in $U$ incident to a vertex in $V$ has eigenvector entry 1.  $G^+$ is a pineapple graph, and it is easy to see that there is a single vertex in a pineapple graph with maximum eigenvector entry.  It follows that the vertices in $V$ are all adjacent to the same vertex in $U$, and hence $G$ is a pineapple graph.
\end{proof}


This chapter is based on the papers ``Three conjectures in extremal spectral graph theory'',
 \cite{TaitTobin2017}, to appear in \textit{Journal of Combinatorial Theory, Series B},
 and ``Characterizing graphs of maximum principal ratio'', submitted to
 \textit{Electronic Journal of Linear Algebra} \cite{TaitTobin2015},
both written jointly with Michael Tait.  The dissertation
author was the primary investigator and author of the paper.

