\chapter{Introduction}

\section{Preliminaries}

The subject of this dissertation is \textit{spectral graph theory}, which studies graphs through
various associated matrices, such as the adjacency matrix or normalized Laplacian.
We will address several problems in this area, with a common theme
of computing or maximizing a spectral parameter, such as the spectral radius, over some families of graphs.  
In this section, we provide an overview of the background terminology and results that will be used throughout the
dissertation, and establish notation.  The section concludes with a summary of the main results.


A graph $G$ is a pair $(V,E)$, where $E$ is a set of \textit{vertices} and $E$ is a set of unordered
pairs of vertices, which are called the \textit{edges} of $G$.  For clarity, we will often
use the notation $V = V(G)$ and $E = E(G)$.  A subgraph of $G$ is a graph whose vertex set and
edge set are subsets of $V(G)$ and $E(G)$ respectively.


Two vertices $x,y$ are said to be \textit{adjacent}
if the pair $(x,y)$ belongs to the edge set.  The \textit{neighbors} of a vertex $x$, denoted $N(x)$,
is the set of all vertices that are adjacent to $x$.  The \textit{degree} of a vertex $x$, denoted $d_x$,
is defined by $d_x = |N(x)|$.  The \textit{average degree} $d$ of a graph is then given by
\[ d = \sum_{x \in V(G)} d_x = \frac{2 |E(G)|}{|V(G)|}. \]


% TODO: weighted graphs

\section{Spectral Graph Theory}

\subsection{Matrices Associated to Graphs}


Given a graph $G$ on $n$ vertices, many $n \times n$ matrices which encode the structure
of the graph have been studied, including the adjacency matrix $A$, the combinatorial Laplacian $L$,
the normalized Laplacian $\mathcal{L}$, the distance matrix $D$ and the signless Laplacian $Q$.
We will be concerned with three of these, the adjacency matrix, and the combinatorial and normalized
Laplacians.  In this subsection we define these matrices, and discuss some of their properties.
Throughout this subsection we will be considering a graph $G$ with vertex set
$V(G) = \left\{ 1,2, \cdots, n\right\}$.


The adjacency matrix is the $n \times n$ matrix defined by
\[
 A(i,j) =
  \begin{cases} 
      \hfill 1 \hfill & \text{if $(i,j)$ is an edge of $G$} \\
      \hfill 0 \hfill & \text{if $(i,j)$ is not an edge of $G$.} \\
  \end{cases}
\]
This is a symmetric matrix, and so will have $n$ real eigenvalues and a basis of
$n$ orthogonal eigenvectors.  We will denote the eigenvalues of the adjacency matrix
by $\lambda_1 \geq \lambda_2 \geq \cdots \geq \lambda_n$.  By the Perron--Frobenius theorem,
if the graph $G$ is connected then $\lambda_1 > \lambda_2$, and we can choose
eigenvector corresponding to $\lambda_1$ whose entries are all strictly positive.






\subsection{Fundamental Inequalities}

\subsection{Graph Coverings and Projections}

\section{Overview of Results}

%\section{A section}
%Lorem ipsum dolor sit amet, consectetuer adipiscing elit. Nulla odio
%sem, bibendum ut, aliquam ac, facilisis id, tellus. Nam posuere pede
%sit amet ipsum. Etiam dolor. In sodales eros quis pede.  Quisque sed
%nulla et ligula vulputate lacinia. In venenatis, ligula id semper
%feugiat, ligula odio adipiscing libero, eget mollis nunc erat id orci.
%Nullam ante dolor, rutrum eget, vestibulum euismod, pulvinar at, nibh.
%In sapien. Quisque ut arcu. Suspendisse potenti. Cras consequat cursus
%nulla.

%\subsection{A Figure Example}
%\label{ssec:figure_example}

%This subsection shows a sample figure.

%\begin{figure}[h] 
%  \centering
%  \includegraphics[width=0.5\textwidth]{sandiego}
%  \caption[Short figure caption (must be \protect{$< 4$} lines in the list of figures)]{A picture of San Diego.  Note that figures must be on their own line (no neighboring text) and captions must be single-sp%aced and appear \protect\textit{below} the figure.  Captions can be as long as you want, but if they are longer than 4 lines in the list of figures, you must provide a short figure caption.\index{SanDiego}} 
%  \label{fig:sandiego}
%\end{figure}

%\subsection{A Table Example}

%While in Section \ref{ssec:figure_example} Figure \ref{fig:sandiego} we had a majestic figure, here we provide a crazy table example.


%%%% TABLE 1 %%%%
%\vspace{0.25in}
%\begin{table}[!ht]
%\caption[Short figure caption (must be \protect{$< 4$} lines in the list of tables)]{A table of when I get hungry.  Note that tables must be on their own line (no neighboring text) and captions must be single-spaced and appear \protect\textit{above} the table.  Captions can be as long as you want, but if they are longer than 4 lines in the list of figures, you must provide a short figure caption.}

%\vspace{-0.25in}
%\begin{center}
%\begin{tabular}{|p{1in}|p{2in}|p{3in}|}

%\hline
%Time of day & Hunger Level & Preferred Food \\

%\hline
%8am & high & IHOP (French Toast) \\

%\hline
%noon & medium & Croutons (Tomato Basil Soup \& Granny Smith Chicken Salad) \\

%\hline
%5pm & high & Bombay Coast (Saag Paneer) or Hi Thai (Pad See Ew) \\

%\hline
%8pm & medium & Yogurt World (froyo!) \\

%\hline
%\end{tabular}
%\end{center}
%\label{tab:analysis3}
%\end{table}


