\chapter{Introduction}

\section{Preliminaries}

The subject of this dissertation is \textit{spectral graph theory}, which studies graphs through
various associated matrices, such as the adjacency matrix or normalized Laplacian.
We will address several problems in this area, with a common theme
of computing or maximizing a spectral parameter, such as the spectral radius, over some families of graphs.  
In this section, we provide an overview of the background terminology and results that will be used throughout the
dissertation, and establish notation.  The section concludes with a summary of the main results.


A graph $G$ is a pair $(V,E)$, where $E$ is a set of \textit{vertices} and $E$ is a set of unordered
pairs of vertices, which are called the \textit{edges} of $G$.  For clarity, we will often
use the notation $V = V(G)$ and $E = E(G)$.  A subgraph of $G$ is a graph whose vertex set and
edge set are subsets of $V(G)$ and $E(G)$ respectively.


Two vertices $x,y$ are said to be \textit{adjacent}
if the pair $(x,y)$ belongs to the edge set.  The \textit{neighbors} of a vertex $x$, denoted $N(x)$,
is the set of all vertices that are adjacent to $x$.  The \textit{degree} of a vertex $x$, denoted $d_x$,
is defined by $d_x = |N(x)|$.  The \textit{average degree} $d$ of a graph is then given by
\[ d = \sum_{x \in V(G)} d_x = \frac{2 |E(G)|}{|V(G)|}. \]
A graph is \textit{$d$-regular} if every vertex has degree $d$.


% TODO: weighted graphs

\section{Spectral Graph Theory}

\subsection{Matrices Associated to Graphs}


Given a graph $G$ on $n$ vertices, many $n \times n$ matrices which encode the structure
of the graph have been studied, including the adjacency matrix $A$, the combinatorial Laplacian $L$,
the normalized Laplacian $\mathcal{L}$, the distance matrix $D$ and the signless Laplacian $Q$.
We will be concerned with three of these, the adjacency matrix, and the combinatorial and normalized
Laplacians.  In this subsection we define these matrices, and discuss some of their properties.
Throughout this subsection we will be considering a graph $G$ with vertex set
$V(G) = \left\{ 1,2, \cdots, n\right\}$.


The adjacency matrix is the $n \times n$ matrix defined by
\[
 A(i,j) =
  \begin{cases} 
      \hfill 1 \hfill & \text{if $(i,j)$ is an edge of $G$} \\
      \hfill 0 \hfill & \text{if $(i,j)$ is not an edge of $G$.} \\
  \end{cases}
\]
This is a symmetric matrix, and so will have $n$ real eigenvalues and a basis of
$n$ orthogonal eigenvectors.  We will denote the eigenvalues of the adjacency matrix
by $\lambda_1 \geq \lambda_2 \geq \cdots \geq \lambda_n$.  By the Perron--Frobenius theorem,
if the graph $G$ is connected then $\lambda_1 > \lambda_2$, and we can choose
eigenvector corresponding to $\lambda_1$ whose entries are all strictly positive.


%TODO: Courant-Fisher



\subsection{Fundamental Inequalities}

We will frequently manipulate inequalities involving properties of the graph, such as the number
of edges, and spectral parameters, such as the eigenvalues and eigenvector entries.
We introduce such manipulations by giving the proofs of some classic inequalities in spectral graph
theory which we will use.


Throughout this section we will assume that $\mathbf{v}$ is the eigenvector
corresponding to the eigenvalue $\lambda_1$ of the adjacency matrix.  Additionally we assume that
it is normalized so that it has maximum entry equal to 1, and that vertex $x$ is a vertex attaining this.  F
rom the definition of the adjacency matrix, we have
\begin{equation}
  \lambda_1 \mathbf{v}_y = \sum_{z \sim y} \mathbf{v}_z .
\end{equation}
Our choice of normalization yields
\begin{equation}
  \lambda_1 \mathbf{v}_y = \sum_{y \sim x} \mathbf{v}_y\leq d_y .
\end{equation}


\begin{theorem}[Stanley \cite{Stanley1987}]
  Given a graph $G$ with $m$ edges and largest adjacency eigenvalue $\lambda_1$, we have
  \[ \lambda_1 \leq \frac{-1 + \sqrt{8m + 1}}{2} ,\]
  with equality only when $G$ is the union of a complete graph and a (possibly empty) independent set of vertices.
\end{theorem}
\begin{proof}
  We have
  \begin{eqnarray*}
    \lambda_1^2 = \lambda_1^2 \mathbf{v}_x & = & \sum_{y \sim x} \lambda_1 \mathbf{v}_y \leq \sum_{y \sim x} d_y \leq 2m - d_x \leq 2m - \lambda_1, 
  \end{eqnarray*}
  where the last inequality follows from $\lambda_1 \leq d_x$.  This implies the desired inequality.  For the equality case, we must have
  $\lambda_1 = d_x$ and $\sum_{y \sim x} d_y = 2m - d_x$.  This happens only if every edge in $G$ is incident to a vertex in $N(x)$,
  and if there is a $d_x$-regular connected component.  This is exactly the union of a complete graph and an independent set of vertices. 
  
\end{proof}


There have been many generalizations of this result \cite{Friedland1988, Hong1988, HongEtAl2001, Nikiforov2002, DasKumar2004},
as well as several
similar inequalities, which incorporate additional information about the graph structure
such as the minimum or maximum degree \cite{FavaronMaheoSacle1993, BermanZhang2001, Nikiforov2006Walks}.  We highlight one of these generalizations here, which we will have occasion to use later.

\begin{theorem}[Hong \cite{Hong1988}]
  Given a graph $G$ with $m$ edges and largest adjacency eigenvalue $\lambda_1$, we have
  \[ \lambda_1 \leq \sqrt{2m - n + 1} ,\]
  with equality only when $G$ is either $K_n$ or $K_{1,n}$.
\end{theorem}
\noindent This can be proved with a small modification to the proof above.  


\section{Overview of Results}

The remainder of this thesis is broken into three chapters.  In Chapter~2 we consider
measures of graph irregularity, which are statistics that quantify how much a graph deviates
from being regular.
In particular, we consider two such statistics. The first is the principal ratio, which,
for a connected graph, is the ratio of the largest and smallest entries of the leading
eigenvector of the adjacency matrix.  For this statistic, we characterze the extremal
graphs for large enough $n$.  This was conjectured by Cioab\u{a} and Gregory
\cite{CioabaGregory2007}.
\begin{theorem}\label{main_theorem}
  For sufficiently large $n$, the connected graph $G$ on $n$
  vertices with largest principal ratio is a kite graph.
\end{theorem}
\noindent The second graph irregularity measure we consider is  $\lambda_1 - d$,
the difference between the largest adjacency eigenvalue and the average degree.
In this case we show that the extremal connected graphs are \textit{pineapple graphs}, which
are a clique with a set of pendant edges added to a single vertex on the clique.
This was conjectured by Aouchiche et. al. \cite{AouchicheEtAl2008}.
\begin{theorem}
  For sufficiently large $n$, the connected graph that maximizes $\lambda_1 - d$
  is a pineapple graph.
\end{theorem}



In Chapter~3, we consider planar and outerplanar graphs, and maximize the adjacency
spectral radius over each of these families for a fixed number of vertices.  The problem
of determining the maximum spectral radius over some family of geometric graphs has been
studied by many authors, with the case of planar graphs receiving particular attention.
%TODO: CITATIONS
We show that for sufficiently large $n$ the extremal graphs are the graphs $P_2 + P_{n-2}$.
This was conjectured independently by Boots and Royle in 1991 \cite{BootsRoyle1991} and
by Cao and Vince in 1993 \cite{CaoVince1993}.
\begin{theorem}
  For sufficiently large $n$, the planar graph on $n$ vertices that maximizes $\lambda_1$
  is $P_2 + P_{n-2}$.
\end{theorem}
\noindent We also answer the analogous problem for outerplanar graphs, showing that
the extremal graph in that case is given by $P_1 + P_{n-1}$, as conjectured
by Cvetkovi\'{c} and Rowlinson \cite{CvetkovicRowlinson1990}.
\begin{theorem}
  For sufficiently large $n$, the planar graph on $n$ vertices that maximizes $\lambda_1$
  is $P_1 + P_{n-1}$.
\end{theorem}


Finally, in Chapter~4 we study several families of graphs that arise in the context
of \textit{substring reversals}.  Given a permutation $\sigma$ written in list notation
$(\sigma_1, \sigma_2, \cdots, \sigma_n)$, a substring reversal is any permutation
of the form
\[ (\sigma_1, \sigma_2, \cdots, \sigma_{i-1}, \sigma_j, \sigma_{j-1}, \cdots, \sigma_{i}, \sigma_{j+1}, \cdots, \sigma_n) .\]
When $i = 1$, this is a \textit{prefix reversal}.  The \textit{reversal graph} has vertex set $S_n$
and where two vertices are adjacent if they can be obtained from one another by a substring reversal.
The study of this graph is motivated by applications in mathematical biology \cite{BafnaPevzner1996}.
Our main result is determining the spectral gap of this graph.
%TODO: more citations
\begin{theorem}
  Let $\lambda_1, \lambda_2$ be the largest and second largest eigenvalues
  of the adjacency matrix of the reversal graph.  Then $\lambda_1 - \lambda_2 = n$.
\end{theorem}
\noindent Additionally, we construct a family of graphs which generalize the prefix reversal graph,
for which every graph has spectral gap equal to one.  This provides a combinatorial
proof that the spectral gap of the prefix reversal gap is one, which was proved via
representation theory by Cesi \cite{Cesi2009}, who in turn was answering a question
posed by Gunnells, Scott and Walden \cite{GunnellsEtAl2007}.






%\section{A section}
%Lorem ipsum dolor sit amet, consectetuer adipiscing elit. Nulla odio
%sem, bibendum ut, aliquam ac, facilisis id, tellus. Nam posuere pede
%sit amet ipsum. Etiam dolor. In sodales eros quis pede.  Quisque sed
%nulla et ligula vulputate lacinia. In venenatis, ligula id semper
%feugiat, ligula odio adipiscing libero, eget mollis nunc erat id orci.
%Nullam ante dolor, rutrum eget, vestibulum euismod, pulvinar at, nibh.
%In sapien. Quisque ut arcu. Suspendisse potenti. Cras consequat cursus
%nulla.

%\subsection{A Figure Example}
%\label{ssec:figure_example}

%This subsection shows a sample figure.

%\begin{figure}[h] 
%  \centering
%  \includegraphics[width=0.5\textwidth]{sandiego}
%  \caption[Short figure caption (must be \protect{$< 4$} lines in the list of figures)]{A picture of San Diego.  Note that figures must be on their own line (no neighboring text) and captions must be single-sp%aced and appear \protect\textit{below} the figure.  Captions can be as long as you want, but if they are longer than 4 lines in the list of figures, you must provide a short figure caption.\index{SanDiego}} 
%  \label{fig:sandiego}
%\end{figure}

%\subsection{A Table Example}

%While in Section \ref{ssec:figure_example} Figure \ref{fig:sandiego} we had a majestic figure, here we provide a crazy table example.


%%%% TABLE 1 %%%%
%\vspace{0.25in}
%\begin{table}[!ht]
%\caption[Short figure caption (must be \protect{$< 4$} lines in the list of tables)]{A table of when I get hungry.  Note that tables must be on their own line (no neighboring text) and captions must be single-spaced and appear \protect\textit{above} the table.  Captions can be as long as you want, but if they are longer than 4 lines in the list of figures, you must provide a short figure caption.}

%\vspace{-0.25in}
%\begin{center}
%\begin{tabular}{|p{1in}|p{2in}|p{3in}|}

%\hline
%Time of day & Hunger Level & Preferred Food \\

%\hline
%8am & high & IHOP (French Toast) \\

%\hline
%noon & medium & Croutons (Tomato Basil Soup \& Granny Smith Chicken Salad) \\

%\hline
%5pm & high & Bombay Coast (Saag Paneer) or Hi Thai (Pad See Ew) \\

%\hline
%8pm & medium & Yogurt World (froyo!) \\

%\hline
%\end{tabular}
%\end{center}
%\label{tab:analysis3}
%\end{table}


