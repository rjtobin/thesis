%
%
% UCSD Doctoral Dissertation Template
% -----------------------------------
% http://ucsd-thesis.googlecode.com
%
%


%% REQUIRED FIELDS -- Replace with the values appropriate to you

% No symbols, formulas, superscripts, or Greek letters are allowed
% in your title.
\title{Extremal Spectral Invariants of Graphs}

\author{Robin Joshua Tobin}
\degreeyear{\the\year}

% Master's Degree theses will NOT be formatted properly with this file.
\degreetitle{Doctor of Philosophy}

\field{Mathematics}
%\specialization{Graph theory}  % If you have a specialization, add it here

\chair{Professor Fan Chung Graham}
% Uncomment the next line iff you have a Co-Chair
\cochair{Professor Jacques Verstra\"{e}te}
%
% Or, uncomment the next line iff you have two equal Co-Chairs.
%\cochairs{Professor Chair Masterish}{Professor Chair Masterish}

%  The rest of the committee members  must be alphabetized by last name.
\othermembers{
Professor Ronald Graham\\
Professor Ramamohan Paturi\\
Professor Jeffrey Remmel\\
}
\numberofmembers{5} % |chair| + |cochair| + |othermembers|


%% START THE FRONTMATTER
%
\begin{frontmatter}

%% TITLE PAGES
%
%  This command generates the title, copyright, and signature pages.
%
\makefrontmatter

%% DEDICATION
%
%  You have three choices here:
%    1. Use the ``dedication'' environment.
%       Put in the text you want, and everything will be formated for
%       you. You'll get a perfectly respectable dedication page.
%
%
%    2. Use the ``mydedication'' environment.  If you don't like the
%       formatting of option 1, use this environment and format things
%       however you wish.
%
%    3. If you don't want a dedication, it's not required.
%
%
\begin{dedication}
  To my family (be they Tobins, Dohertys or Ryans).
\end{dedication}


% \begin{mydedication} % You are responsible for formatting here.
%   \vspace{1in}
%   \begin{flushleft}
% 	To me.
%   \end{flushleft}
%
%   \vspace{2in}
%   \begin{center}
% 	And you.
%   \end{center}
%
%   \vspace{2in}
%   \begin{flushright}
% 	Which equals us.
%   \end{flushright}
% \end{mydedication}



%% EPIGRAPH
%
%  The same choices that applied to the dedication apply here.
%
\begin{epigraph} % The style file will position the text for you.
  \emph{Shut up!\\
  I am working Cape Race.}\\
  ---Jack Phillips
\end{epigraph}

% \begin{myepigraph} % You position the text yourself.
%   \vfil
%   \begin{center}
%     {\bf Think! It ain't illegal yet.}
%
% 	\emph{---George Clinton}
%   \end{center}
% \end{myepigraph}


%% SETUP THE TABLE OF CONTENTS
%
\tableofcontents
\listoffigures  % Comment if you don't have any figures
%\listoftables   % Comment if you don't have any tables



%% ACKNOWLEDGEMENTS
%
%  While technically optional, you probably have someone to thank.
%  Also, a paragraph acknowledging all coauthors and publishers (if
%  you have any) is required in the acknowledgements page and as the
%  last paragraph of text at the end of each respective chapter. See
%  the OGS Formatting Manual for more information.
%
\begin{acknowledgements}

Firstly, I thank Fan Chung and Jacques Verstra\"{e}te, for providing a seemingly endless
stream of interesting problems.  Every problem that I could not tackle introduced me
to some new technique or idea, and without these none of the work in this thesis would have
been possible.  I thank them both for their patience, and their guidance.  I thank all
of the members of my thesis committee for their time over the last several years.


I am indebted to Vladimir Dotsenko, Conor Houghton and David Malone, for introducing me
to many interesting things, both within mathematics and outside of it.
%Most people are not lucky enough to have one such person in their early academic lives,
%to have known all three was beyond a privilege.
I thank Mike Tait for a collaboration that I enjoyed greatly.
Vladimir Nikiforov, Xing Peng, and an anonymous referee have provided
valuable feedback on some of the papers that this thesis is based upon.


Despite my best efforts to the contrary, a number of people have made San Diego feel
like home over the last few years.  I especially thank Leonard Haff; Kim; everyone
from 5412 especially Sinan, Frankie, Sebastian, David, Will;  Lyla, Brian, Mike, Rob, Jay,
Hooman; my academic siblings, especially Franklin and Mark. 


Lastly, I thank my family, especially my parents (all four of them) and my siblings.
Five years is a long time, and I hope that wherever I am
in the future I will at least be closer to home.


Chapters 2 and 3 are based on the papers ``Three conjectures in extremal spectral graph theory'',
 \cite{TaitTobin2017}, to appear in \textit{Journal of Combinatorial Theory, Series B},
 and ``Characterizing graphs of maximum principal ratio'', submitted to
 \textit{Electronic Journal of Linear Algebra} \cite{TaitTobin2015},
 both written jointly with Michael Tait.
Chapter 4 is based on the paper ``The Spectral Gap of Graphs Arising from Substring Reversals'',
submitted to \textit{Journal of Combinatorics}, written jointly with Fan Chung.

\end{acknowledgements}


%% VITA
%
%  A brief vita is required in a doctoral thesis. See the OGS
%  Formatting Manual for more information.
%
\begin{vitapage}
\begin{vita}
  \item[2011] B.~A. in Mathematics, The University of Dublin, Trinity College.
  \item[2013] M.~A. in Mathematics, University of California, San Diego.
  \item[2017] Ph.~D. in Mathematics, University of California, San Diego.
\end{vita}
%\begin{publications}
%  \item Your Name, ``A Simple Proof Of The Riemann Hypothesis'', \emph{Annals of Math}, 314, 2007.
%  \item Your Name, Euclid, ``There Are Lots Of Prime Numbers'', \emph{Journal of Primes}, 1, 300 B.C.
%\end{publications}
\end{vitapage}


%% ABSTRACT
%
%  Doctoral dissertation abstracts should not exceed 350 words.
%   The abstract may continue to a second page if necessary.
%
\begin{abstract}
 We address several problems in spectral graph theory, with a common theme of
 optimizing or computing a spectral graph invariant, such as the spectral radius or
 spectral gap, over some family of graphs.  In particular, we study
 measures of graph irregularity, we bound the adjacency spectral radius
 over all outerplanar and planar graphs, and finally we determine the spectral
 gap of reversal graphs and a family of graphs that generalize the
 prefix reversal graph.


 Firstly we study two measures of graph irregularity, the principal ratio and the
 difference between the spectral radius of the adjacency matrix and the average
 degree.  For the principal ratio, we show that 
 the graphs which maximize this
 statistic are the \textit{kite graphs}, which are a clique with a pendant path,
 when the number of vertices is sufficiently large.
 This answers a conjecture of Cioab\u{a} and Gregory.  For the second graph
 irregularity measure, we show that the connected graphs which maximize it are
 \textit{pineapple graphs}, answering a conjecture of Aouchiche et al.


 Secondly we investigate the maximum spectral radius of the adjacency matrix over
 all graphs on $n$ vertices within certain well-known graph families.
 Our main result is showing that the planar graph on $n$ vertices with maximal adjacency
 spectral radius is the join $P_2 + P_{n-2}$, when $n$ is sufficiently large.  This
 was conjectured by Boots and Royle.  Additionally, we
 identify the outerplanar graph with maximal spectral radius, answering a conjecture
 of Cvetkovi\`{c} and Rowlinson.


 Finally, we determine the spectral gap of various Cayley graphs of the symmetric group
 $S_n$, which arise in the context of substring reversals.  This includes an elementary
 proof that the prefix reversal (or \textit{pancake flipping} graph) has spectral
 gap one, originally proved via representation theory by Cesi.  We generalize this
 by showing that a large family of related graphs all have unit spectral gap.

\end{abstract}


\end{frontmatter}
